\documentclass{article}
\usepackage{amsmath}
\usepackage[mathletters]{ucs}
\usepackage[utf8x]{inputenc}
\usepackage[margin=1.5in]{geometry}
\usepackage{enumerate}
\newtheorem{theorem}{Theorem}
\usepackage[dvipsnames]{xcolor}
\usepackage{pgfplots}
\setlength{\parindent}{0cm}
\usepackage{graphics}
\usepackage{graphicx} % Required for including images
\usepackage{subcaption}
\usepackage{bigintcalc}
\usepackage{pythonhighlight} %for pythonkode \begin{python}   \end{python}
\usepackage{appendix}
\usepackage{arydshln}
\usepackage{physics}
\usepackage{tikz-cd}
\usepackage{booktabs} 
\usepackage{adjustbox}
\usepackage{mdframed}
\usepackage{relsize}
\usepackage{physics}
\usepackage[thinc]{esdiff}
\usepackage{fixltx2e}
\usepackage{esint}  %for lukket-linje-integral
\usepackage{xfrac} %for sfrac
\usepackage[colorlinks=true]{hyperref} %for linker, må ha med hypersetup
\usepackage[noabbrev, nameinlink]{cleveref} % to be loaded after hyperref
\usepackage{amssymb} %\mathbb{R} for reelle tall, \mathcal{B} for "matte"-font
\usepackage{listings} %for kode/lstlisting
\usepackage{verbatim}
\usepackage{graphicx,wrapfig,lipsum,caption} %for wrapping av bilder
\usepackage{mathtools} %for \abs{x}
\usepackage[norsk]{babel}
\definecolor{codegreen}{rgb}{0,0.6,0}
\definecolor{codegray}{rgb}{0.5,0.5,0.5}
\definecolor{codepurple}{rgb}{0.58,0,0.82}
\definecolor{backcolour}{rgb}{0.95,0.95,0.92}
\lstdefinestyle{mystyle}{
    backgroundcolor=\color{backcolour},   
    commentstyle=\color{codegreen},
    keywordstyle=\color{magenta},
    numberstyle=\tiny\color{codegray},
    stringstyle=\color{codepurple},
    basicstyle=\ttfamily\footnotesize,
    breakatwhitespace=false,         
    breaklines=true,                 
    captionpos=b,                    
    keepspaces=true,                 
    numbers=left,                    
    numbersep=5pt,                  
    showspaces=false,                
    showstringspaces=false,
    showtabs=false,                  
    tabsize=2
}

\lstset{style=mystyle}
\author{Oskar Idland}
\title{Oblig 9}
\date{}
\begin{document}        
\maketitle
\newpage

\section*{Problem 9.2 H}
We know that for any state the upper bound for the ground state energy is given by
\[
E_{\text{gs}} ≤ \frac{\bra{ψ}H\ket{ψ}}{\bra{ψ}\ket{ψ}}
\]
I guess a state of the form:
\[
\ket{ψ(x)} Ae^{-b x^2}
\]
As the potential would trap the particle around the origin. First we must normalize the state:
\[
\int_{-\infty}^{\infty} ψ^{*} ψ \ dx = A^2 ∫_{-∞}^{∞} e^{-2bx^2} \ \mathrm{d}x = 1
\]
Using Rottman we get:
\[
A^2 = \sqrt{\frac{π}{2b}} → A = \sqrt[4]{\frac{π}{2b}}
\]
With this normalized state we can calculate the upper bound for the ground state energy using:
\[
E_{\text{gs}} \le   \bra{ψ}\hat{H}\ket{ψ}
\]

Next we must find the Hamiltonian given by:
\[
\hat{H} = -\frac{ℏ^2}{2m} \frac{∂^2 }{∂ x^2} + V(x)
\]
Then let it act on our state:

\[
\hat{H} \ket{ψ(x)} = -\frac{ℏ^2}{2m} \frac{∂^2 }{∂ x^2} \ket{ψ(x)} + V(x) \ket{ψ(x)}
\]
\[
\hat{H} \ket{ψ(x)} = -\frac{ℏ^2}{2m} \left( -2bAe^{-bx^2} + 4bx^2e^{-bx^2} \right) + α \left|x\right| A e^{-bx^2}
\]
\[
\hat{H} \ket{ψ(x)}  = \frac{ℏ^2}{m} bAe^{-bx^2} - \frac{ℏ^2}{m} 2bAx^2e^{-bx^2} + α \left|x\right| A e^{-bx^2}
\]
The expectation value of the Hamiltonian is then given by:
\[
\bra{ψ(x)}\hat{H}\ket{ψ(x)} = \frac{ℏ^2}{m} bA^2 \underbrace{∫_{-∞}^{∞} e^{-2bx^2} \ dx}_{\text{Term 1}} - \frac{ℏ^2}{m} 2bA^2 \underbrace{∫_{-∞}^{∞} x^2e^{-2bx^2} \ dx}_{\text{Term 2}} + α A^2 \underbrace{∫_{-∞}^{∞} \left|x\right| e^{-2bx^2} \ dx}_{\text{Term 3}}
\]
\subsection*{Term 1}
We have already normalized this term previously and therefore get:
\[
\underline{\frac{ℏ^2}{m}bA}
\]

\subsection*{Term 2}
Using Rottman we find that this integral is evaluated to the following:
\[
∫_{-∞}^{∞} x^2 e^{-2bx^2} \ \mathrm{d}x = \frac{1}{2} \sqrt{\frac{π}{8b^3}}
\]
This gives the final:
\[
\underline{-\frac{ℏ^2}{m}A\sqrt{\frac{π}{8b^3}}}
\]

\subsection*{Term 3}
Splitting up the integral into two parts we get:
\[
∫_{-∞}^{∞} \left|x\right|e^{-2bx^2} \ \mathrm{d}x = ∫_{-∞}^{0} -xe^{-2bx^2} \ \mathrm{d}x + ∫_{0}^{∞} xe^{-2bx^2} \ \mathrm{d}x
\]
Which can be rewritten as:
\[
2 ∫_{0}^{∞} xe^{-2bx^2} \ \mathrm{d}x
\]
Again using Rottman we find this to be:
\[
\frac{1}{2b}
\]
Adding the constants we get:
\[
\underline{αA^2 \frac{1}{2b}}
\]
\subsection*{Final result}
\[
\left<\hat{H}\right> = \frac{ℏ^2}{m}bA - \frac{ℏ^2}{m}A\sqrt{\frac{π}{8b^3}} + αA^2 \frac{1}{2b}
\]



\section*{Problem 9.3 (H)}
We try the same state as in the previous exercise:
\[
\ket{ψ(x,y,z)} = A e^{-r ^2 / 2L} = A e^{-(x^2+y^2+z^2)/2L}
\]


Again we must normalize:
\[
∫_{-∞}^{∞} \left|ψ\right|^2 \ \mathrm{d}r = A^2 ∫_{-∞}^{∞} ∫_{-∞}^{∞} ∫_{-∞}^{∞} e^{-(x^2 + y^2 + z^2)} \ \mathrm{d}x \ \mathrm{d}y \ \mathrm{d}z = 1
\]
This is just the same integral repeated. Using Rottmann we find that this is equal to:
\[
A^2 \left( \sqrt{πL^2} \right)^3 = 1 → A = \frac{1}{\left(πL^2\right)^{3 / 4}} = \left(πL^2\right)^{- 3 / 4}
\]

For simplicity, we divide the Hamiltonian into a kinetic and potential part:
\[
\hat{H} = \hat{T} + \hat{V}  → \left<\hat{H}\right> = \left<\hat{T}\right> + \left<\hat{V}\right>
\]

\[
\left<\hat{T}\right> = - \frac{ℏ^2}{2m} \left|A\right|^2 ∫_{-∞}^{∞} ∫_{-∞}^{∞} ∫_{-∞}^{∞} \bra{ψ} ∇^2 \ket{ψ} \ \mathrm{d}x \ \mathrm{d}y \ \mathrm{d}z
\]
The laplacian yields the following:
\[
∇^2 \ket{ψ} = \left(-\frac{3}{L^2} + \frac{x^2 + y^2 + z^2}{L^{4}}\right) \ket{ψ}
\]
The integral is the same as before and we get:
\[
\left<\hat{T}\right> = \frac{3ℏ^2}{4mL^2}
\]

The potential part is quite straight forward as we only have to evaluate the potential at the origin:

\[
\left<\hat{V}\right> = -α\left|A\right|^2 ∫_{-∞}^{∞} ∫_{-∞}^{∞} ∫_{-∞}^{∞} \left|ψ\right|^2 δ^3(\vec{r}) \ \mathrm{d}x \ \mathrm{d}y \ \mathrm{d}z = α\left|A\right|^2
\]
\[
\left<\hat{V}\right> = -α\left|A\right|^2 = -α\left(πL^2\right)^{- 3 / 2}
\]

Adding both together we get the expectation value of the Hamiltonian:
\[
\left<\hat{H}\right> = \frac{3ℏ^2}{4mL^2} - α\left(πL^2\right)^{- 3 / 2}
\]

We see that the expectation value diverges to infinity as $L → 0$ 


\end{document}