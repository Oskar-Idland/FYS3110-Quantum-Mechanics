\documentclass{article}
\usepackage{amsmath}
\usepackage[mathletters]{ucs}
\usepackage[utf8x]{inputenc}
\usepackage[margin=1.5in]{geometry}
\usepackage{enumerate}
\newtheorem{theorem}{Theorem}
\usepackage[dvipsnames]{xcolor}
\usepackage{pgfplots}
\pgfplotsset{compat=1.18}
\setlength{\parindent}{0cm}
\usepackage{graphics}
\usepackage{graphicx} % Required for including images
\usepackage{subcaption}
\usepackage{bigintcalc}
\usepackage{pythonhighlight} %for pythonkode \begin{python}   \end{python}
\usepackage{appendix}
\usepackage{arydshln}
\usepackage{physics}
\usepackage{booktabs} 
\usepackage{adjustbox}
\usepackage{mdframed}
\usepackage{relsize}
\usepackage{physics}
\usepackage[thinc]{esdiff}
\usepackage{esint}  %for lukket-linje-integral
\usepackage{xfrac} %for sfrac
\usepackage{hyperref} %for linker, må ha med hypersetup
\usepackage[noabbrev, nameinlink]{cleveref} % to be loaded after hyperref
\usepackage{amssymb} %\mathbb{R} for reelle tall, \mathcal{B} for "matte"-font
\usepackage{listings} %for kode/lstlisting
\usepackage{verbatim}
\usepackage{graphicx,wrapfig,lipsum,caption} %for wrapping av bilder
\usepackage{mathtools} %for \abs{x}
\usepackage[norsk]{babel}
\usepackage{cancel}
\definecolor{codegreen}{rgb}{0,0.6,0}
\definecolor{codegray}{rgb}{0.5,0.5,0.5}
\definecolor{codepurple}{rgb}{0.58,0,0.82}
\definecolor{backcolour}{rgb}{0.95,0.95,0.92}
\lstdefinestyle{mystyle}{
    backgroundcolor=\color{backcolour},   
    commentstyle=\color{codegreen},
    keywordstyle=\color{magenta},
    numberstyle=\tiny\color{codegray},
    stringstyle=\color{codepurple},
    basicstyle=\ttfamily\footnotesize,
    breakatwhitespace=false,         
    breaklines=true,                 
    captionpos=b,                    
    keepspaces=true,                 
    numbers=left,                    
    numbersep=5pt,                  
    showspaces=false,                
    showstringspaces=false,
    showtabs=false,                  
    tabsize=2
}

\lstset{style=mystyle}
\author{Oskar Idland}
\title{Oblig 7}
\date{}
\begin{document}
\maketitle
\newpage

\section*{Problem 1}
\subsection*{a)}
We know that
\[
\vec{J} = \vec{L} + \vec{S} → \vec{J}^2 = (\vec{L} + \vec{S})^2 = \vec{L}^2 + \vec{S}^2 + 2\vec{L} ⋅ \vec{S}
\]
We therefore know that:
\[
\vec{L} ⋅ \vec{S} = \frac{1}{2}(\vec{J}^2 - \vec{L}^2 - \vec{S}^2)
\]
The eigenvalues will then have the same values
We begin with $S^2$
\[
S^2 = ℏ^2 s(s+1)  = ℏ^2 \frac{3}{4}
\]
Next we have $L^2$:
\[
L^2 = ℏ^2 l(l+1) = 6ℏ^2
\]
We have $j$-values in integer steps in the range:
\[
\left|l-s\right| \le j \le l+s → \frac{3}{2} \le  j \le \frac{5}{2}   
\]
Therefore we know $j$ can be either $j_1 = 5 / 2$ or $j_2 = 3 / 2$
Next we have $J^2$:
    \[
J_1^2 = ℏ^2 j_1(j_1+1) = ℏ^2 \frac{35}{4}
\]
\[
J_2^2 = ℏ^2 j_2(j_2+1) = ℏ^2 \frac{15}{4}
\]
Then we compute the eigenvalues of the dot product for $j_1$:
\[
\operatorname{eig} \left(\mathbf{L} ⋅ \mathbf{S}\right) = \frac{1}{2} \left(ℏ^2 \frac{35}{4} - 6ℏ^2 -  \frac{3}{4}ℏ^2\right) = ℏ^2
\]
And for $j_2$:
\[
\operatorname{eig} \left(\mathbf{L} ⋅ \mathbf{S}\right) = \frac{1}{2} \left(ℏ^2 \frac{15}{4} - 6ℏ^2 -  \frac{3}{4}ℏ^2\right) = - ℏ^2 \frac{3}{2}
\]
Therefore we know:
\[
E_{so_{j = 5 / 2}} = λ \quad , \quad E_{so_{j = 3 / 2}} = - \frac{3}{2} λ
\]
The full Hamiltonian is then dependent on the $j$-value:
\[
E_j = E_3 + E_{so_j} 
\]
Adding in the energy eigenvalues we get:
\[
E_j = \frac{E_1}{3^2} + E_{so_j}
\]
Where 


There are two possible values for the $m_s$, and five possible values for $m_l$. If $j = 3 / 2$ then there are  $2j + 1 = 4$ possible values for $m_j$ meaning a total degeneracy of 40. If $j = 5 / 2$ there are $2j + 1 = 6$ possible values giving a total degeneracy of 60.

\subsection*{b)}
We know $n=3$ and $l=2$. From our previous calculations we know that the lowest energy has $j = 3 / 2$ and $m_j = +1 / 2$. With $l = 2$. As we want to go from describing our state in terms of the total angular momentum to the orbital and spin angular momentum we use the Clebsch-Gordan coefficients for angular momentum 2 and spin 1/2. We therefore have:

\[
\ket{3 / 2,\  +1 / 2} =  \sqrt{\frac{3}{5}} \ket{1,\ -1 / 2} - \sqrt{\frac{2}{5}} \ket{0,\ +1 / 2}
\]
The probability of measuring $m_s = ℏ / 2 $ is therefore $2 / 5$. 

\subsection*{c)}    
The $\hat{J}_z$ operator has eigenvalues of $ℏm_j$, which in turn gives $\hat{H}_b$ eigenvalues of $-bm_j$ We therefore get further splitting in terms of energy, where we get $2j+1$ values of $m_j$. In our case that gives us 4 values for $j = 3 / 2$ and 6 values for $j = 5 / 2$. 

\subsection*{d)}
We already know the lowest energy belongs to $j = 3 / 2$. As the newly added term has a negative sign, we know the lowest possible energy comes from the only two positive values $m_j = 1 / 2$ and $m_j = 3 / 2$

\section*{7.7 (H)}
To have a fully symmetric system as the bosons are identical, we must have both symmetric or antisymmetric spatial and spin part. We find the states which satisfy this using the Clebsch-Gordan coefficients. We look at the cases where the total spin is either  0 or 2. Starting at the upper left of the table we get:
\paragraph{$J = 2$}
\[
\ket{j=2, m=2} = \ket{1,1}
\]
\[
\ket{j=2, m=1} = \frac{1}{\sqrt{2}} \ket{0,1} + \frac{1}{\sqrt{2}} \ket{1,0}
\]
\[
\ket{j=2, m=0} = \frac{1}{\sqrt{6}}\ket{1,-1} + \sqrt{\frac{2}{3}}\ket{0,0} + \frac{1}{\sqrt{6}}\ket{-1,1}
\]
\[
\ket{j=2, m=-1} = \frac{1}{\sqrt{2}}\ket{0,-1} + \frac{1}{\sqrt{2}}\ket{-1,0}
\]
\[
\ket{j=2, m=-2} = \ket{-1,-1}
\]

\paragraph{$J = 1$}
\[
\ket{1,1} = \frac{1}{\sqrt{2}} \ket{0,1} - \frac{1}{\sqrt{2}} \ket{1,0}
\]
\[
\ket{1,0} = \frac{1}{\sqrt{2}} \ket{1,-1} - \frac{1}{\sqrt{2}} \ket{-1,1}
\]
\[
\ket{1,-1} = \frac{1}{\sqrt{2}} \ket{0,-1} - \frac{1}{\sqrt{2}} \ket{-1,0}
\]

\paragraph{$J = 0$}
\[
\ket{j=0, m=0} = \frac{1}{\sqrt{3}} \ket{1,1} - \frac{1}{\sqrt{3}} \ket{0,0} + \frac{1}{\sqrt{3}} \ket{-1,1}
\]

\section*{7.8 (X)}
The way to check if an operator is time-independent or not, is to see if it commutes with the time-independent Hamiltonian. 
\[
\left[\hat{H}, S^{z}_1\right]
\]
Expanding the coefficients and taking out some factors we get the following:
\[
 \frac{J}{ℏ}i \Big(-S_{1}^{y} S_{2}^{x} + S_{1}^{x} S_{2}^{y}\Big) ≠ 0
\]

\[
\left[\hat{H}, S^{z}_{1}\right] = \frac{J}{ℏ}i \Big(S_{1}^{y} S_{2}^{x} - S_{1}^{x} S_{2}^{y}\Big) ≠ 0
\]
Neither operator commutes, but as they only differ by a sign, we know that $\left(S_1^{z} + S_2^{z}\right)$ must equal zero. Naturally we also know that $\left(S_1^{z} - S_2^{z}\right)$ can't be zero and is therefore also time-dependent. In conclusion: \underline{Only $G_3$ is time-independent}. 


\section*{7.9 (X)}
\subsection*{a)}
We use the basis of position 1 being $(1,0,0)$, position 2 being $(0,1,0)$ and position 3 being $(0,0,1)$. We then have the following Hamiltonian:
\[
H ≃ -g
\begin{pmatrix*}[r]
 0 & 1 & 1 \\
 1 & 0 & 1 \\
 1 & 1 & 0 \\
\end{pmatrix*}
\]

\subsection*{b)}
To be Hermtian we check the following:
\[
R = R^{†}  
\]
\[
\ket{2}\bra{1} + \ket{3}\bra{2} + \ket{1}\bra{3} ≠  \ket{1}\bra{2} + \ket{2}\bra{3} + \ket{3}\bra{1}
\]
To see if it is unitary we check the following:
\[
R^{†} = R^{-1}
\]
This is easy if we do the following:
\[
RR^{†} = I
\]
\[
\Big(\ket{2}\bra{1} + \ket{3}\bra{2} + \ket{1}\bra{3}\Big) \Big(\ket{1}\bra{2} + \ket{2}\bra{3} + \ket{3}\bra{1}\Big)
\]
\[
\ket{2}\bra{2} + \ket{3}\bra{3} + \ket{1}\bra{1} = I
\]
Therefore we know that $R$ is unitary. To check for a symmetric transformation we check if it commutes with the Hamiltonian. Looking at its definition we see that it is $-g(R + R^{†})$. 
\[
\left[H, R\right] = -g \left[R + R^{†}, R\right] 
\]
As $R$ obviously commutes with itself we get:
\[
-g \Big(R^{†}R - R R^{†}\Big) = 0
\]
Therefore we know that $R$ is a symmetric transformation.

\subsection*{c)}
\[
R R = \Big(\ket{2}\bra{1} + \ket{3}\bra{2} + \ket{1}\bra{3}\Big) \Big(\ket{2}\bra{1} + \ket{3}\bra{2} + \ket{1}\bra{3}\Big) = \ket{2}\bra{3} + \ket{3}\bra{1} + \ket{1}\bra{2}
\]
\[
R^2 R = \Big(\ket{2}\bra{3} + \ket{3}\bra{1} + \ket{1}\bra{2}\Big) \Big(\ket{2}\bra{1} + \ket{3}\bra{2} + \ket{1}\bra{3}\Big) 
\]
\[
\ket{2}\bra{2} + \ket{3}\bra{3} + \ket{1}\bra{1} = I
\]
If $R^3 = I$, and $R$ has an eigenvalue $λ_R$ and $I$ has eigenvectors $λ_I = 1$, then we know that $λ_R^3 = 1$. Which means $λ_R =  \sqrt[3]{1} = 1$. We know $R$ is not Hermitian which opens the door for complex values. We know that $e^{i 2πn} = 1$. Therefore, we know that $R$ has eigenvalues $λ_R = e^{i 2πn / 3}$ where $n = 0, 1, 2$. 

\subsection*{d)}
Using the time evolution operator we get:
\[
e^{iHt / ℏ} = e^{-ig(R + R^{†})t / ℏ}
\]
\[
\ket{ψ(t)} = e^{-ig(R + R^{†})t / ℏ} \ket{ψ(0)} 
\]
Can' really do more as I don't have a proper state.

\end{document}