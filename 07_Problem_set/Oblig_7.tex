\documentclass{article}
\usepackage{amsmath}
\usepackage[mathletters]{ucs}
\usepackage[utf8x]{inputenc}
\usepackage[margin=1.5in]{geometry}
\usepackage{enumerate}
\newtheorem{theorem}{Theorem}
\usepackage[dvipsnames]{xcolor}
\usepackage{pgfplots}
\pgfplotsset{compat=1.18}
\setlength{\parindent}{0cm}
\usepackage{graphics}
\usepackage{graphicx} % Required for including images
\usepackage{subcaption}
\usepackage{bigintcalc}
\usepackage{pythonhighlight} %for pythonkode \begin{python}   \end{python}
\usepackage{appendix}
\usepackage{arydshln}
\usepackage{physics}
\usepackage{booktabs} 
\usepackage{adjustbox}
\usepackage{mdframed}
\usepackage{relsize}
\usepackage{physics}
\usepackage[thinc]{esdiff}
\usepackage{esint}  %for lukket-linje-integral
\usepackage{xfrac} %for sfrac
\usepackage{hyperref} %for linker, må ha med hypersetup
\usepackage[noabbrev, nameinlink]{cleveref} % to be loaded after hyperref
\usepackage{amssymb} %\mathbb{R} for reelle tall, \mathcal{B} for "matte"-font
\usepackage{listings} %for kode/lstlisting
\usepackage{verbatim}
\usepackage{graphicx,wrapfig,lipsum,caption} %for wrapping av bilder
\usepackage{mathtools} %for \abs{x}
\usepackage[norsk]{babel}
\usepackage{cancel}
\definecolor{codegreen}{rgb}{0,0.6,0}
\definecolor{codegray}{rgb}{0.5,0.5,0.5}
\definecolor{codepurple}{rgb}{0.58,0,0.82}
\definecolor{backcolour}{rgb}{0.95,0.95,0.92}
\lstdefinestyle{mystyle}{
    backgroundcolor=\color{backcolour},   
    commentstyle=\color{codegreen},
    keywordstyle=\color{magenta},
    numberstyle=\tiny\color{codegray},
    stringstyle=\color{codepurple},
    basicstyle=\ttfamily\footnotesize,
    breakatwhitespace=false,         
    breaklines=true,                 
    captionpos=b,                    
    keepspaces=true,                 
    numbers=left,                    
    numbersep=5pt,                  
    showspaces=false,                
    showstringspaces=false,
    showtabs=false,                  
    tabsize=2
}

\lstset{style=mystyle}
\author{Oskar Idland}
\title{Oblig 7}
\date{}
\begin{document}
\maketitle
\newpage

\section*{Problem 1}
\subsection*{a)}
We know that
\[
\vec{J} = \vec{L} + \vec{S} → \vec{J}^2 = (\vec{L} + \vec{S})^2 = \vec{L}^2 + \vec{S}^2 + 2\vec{L} ⋅ \vec{S}
\]
We therefore know that:
\[
\vec{L} ⋅ \vec{S} = \frac{1}{2}(\vec{J}^2 - \vec{L}^2 - \vec{S}^2)
\]
We begin with $S^2$
\[
S^2 = ℏ^2 s(s+1)  = ℏ^2 \frac{3}{4}
\]
Next we have $L^2$:
\[
L^2 = ℏ^2 l(l+1) = 6ℏ^2
\]
We have $j$-values in integer steps in the range:
\[
\left|l-s\right| \le j \le l+s → \frac{3}{2} \le  j \le \frac{5}{2}   
\]
Therefore we know $j$ can be either $j_1 = 5 / 2$ or $j_2 = 3 / 2$
Next we have $J^2$:
    \[
J_1^2 = ℏ^2 j_1(j_1+1) = ℏ^2 \frac{35}{4}
\]
\[
J_2^2 = ℏ^2 j_2(j_2+1) = ℏ^2 \frac{15}{4}
\]
Then we compute the dot product for $j_1$:
\[
\mathbf{L} ⋅ \mathbf{S} = \frac{1}{2} \left(ℏ^2 \frac{35}{4} - 6ℏ^2 -  \frac{3}{4}ℏ^2\right) = ℏ^2
\]
And for $j_2$:
\[
    \mathbf{L} ⋅ \mathbf{S} = \frac{1}{2} \left(ℏ^2 \frac{15}{4} - 6ℏ^2 -  \frac{3}{4}ℏ^2\right) = - ℏ^2 \frac{3}{2}
\]
Therefore we know:
\[
\hat{H}_{so_{j = 5 / 2}} = λ \quad , \quad \hat{H}_{so_{j = 3 / 2}} = - \frac{3}{2} λ
\]
The full Hamiltonian is then dependent on the $j$-value:
\[
\hat{H}_j = \hat{H}_0 + \hat{H}_{so_j} 
\]
There are two possible values for the $m_s$, and five possible values for $m_l$. If $j = 3 / 2$ then there are  $2j + 1 = 4$ possible values for $m_j$ meaning a total degeneracy of 40. If $j = 5 / 2$ there are $2j + 1 = 6$ possible values giving a total degeneracy of 60.


\end{document}