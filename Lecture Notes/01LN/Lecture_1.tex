\documentclass{article}
\usepackage{amsmath}
\usepackage[mathletters]{ucs}
\usepackage[utf8x]{inputenc}
\usepackage[margin=1.5in]{geometry}
\usepackage{enumerate}
\newtheorem{theorem}{Theorem}
\usepackage[dvipsnames]{xcolor}
\usepackage{pgfplots}
\setlength{\parindent}{0cm}
\usepackage{graphics}
\usepackage{graphicx} % Required for including images
\usepackage{subcaption}
\usepackage{bigintcalc}
\usepackage{pythonhighlight} %for pythonkode \begin{python}   \end{python}
\usepackage{appendix}
\usepackage{arydshln}
\usepackage{physics}
\usepackage{tikz-cd}
\usepackage{booktabs} 
\usepackage{adjustbox}
\usepackage{mdframed}
\usepackage{relsize}
\usepackage{physics}
\usepackage[thinc]{esdiff}
\usepackage{fixltx2e}
\usepackage{esint}  %for lukket-linje-integral
\usepackage{xfrac} %for sfrac
\usepackage{hyperref} %for linker, må ha med hypersetup
\usepackage[noabbrev, nameinlink]{cleveref} % to be loaded after hyperref
\usepackage{amssymb} %\mathbb{R} for reelle tall, \mathcal{B} for "matte"-font
\usepackage{listings} %for kode/lstlisting
\usepackage{verbatim}
\usepackage{graphicx,wrapfig,lipsum,caption} %for wrapping av bilder
\usepackage{mathtools} %for \abs{x}
\usepackage[norsk]{babel}
\definecolor{codegreen}{rgb}{0,0.6,0}
\definecolor{codegray}{rgb}{0.5,0.5,0.5}
\definecolor{codepurple}{rgb}{0.58,0,0.82}
\definecolor{backcolour}{rgb}{0.95,0.95,0.92}
\lstdefinestyle{mystyle}{
    backgroundcolor=\color{backcolour},   
    commentstyle=\color{codegreen},
    keywordstyle=\color{magenta},
    numberstyle=\tiny\color{codegray},
    stringstyle=\color{codepurple},
    basicstyle=\ttfamily\footnotesize,
    breakatwhitespace=false,         
    breaklines=true,                 
    captionpos=b,                    
    keepspaces=true,                 
    numbers=left,                    
    numbersep=5pt,                  
    showspaces=false,                
    showstringspaces=false,
    showtabs=false,                  
    tabsize=2
}

\lstset{style=mystyle}
\author{Oskar Idland}
\title{Bras, Kets and Dirac-Delta}
\date{}
\begin{document}
\maketitle
\newpage

\section{Inner Product}
There are many ways to write the inner product
\begin{itemize}
    \item I. 
    \[
    \left(\ket{u}, \ket{v}\right) = \left(\ket{v}, \ket{u}\right)^{*} 
    \]
    
    \item II. Second linearity makes first linearity impossible. 
    \[
    \left(\ket{u}, α\ket{v_1} + β \ket{v_2}\right) = α\left(\ket{u}, \ket{v_1}\right) + β\left(\ket{u}, \ket{v_1}\right)
    \] 
    \item III. 
    \[
    \left(α \ket{v_1} + β\ket{v_2}, \ket{u}\right) = α^{*} \ket{\ket{v_1}, \ket{u}} + β^{*} \left( \ket{v_2}, \ket{u}\right)
    \] 
    \item IV. 
    \[
    \underbrace{(\ket{v},\ket{v})}_{ℝ} \ge 0 
    \]
\end{itemize}

\subsection{Dirac Notation}
We denote the inner product like this in Dirac-notation. 
\[
(\ket{u}, \ket{v}) = \bra{u}\ket{v} ∈ ℂ
\]
\begin{itemize}
    \item  I 
    \[
    \bra{u}\ket{v} = ( \bra{v}\ket{u})^{*} 
    \] 
    \item II 
    \[
    \ket{v'} = α \ket{v_1} + β \ket{v_2} 
    \]
    
    \item III 
    \[
    \bra{v}\ket{v} \ge 0
    \]
\end{itemize}

\subsection{Representation of Bras}
The bra can be written as an operator operating on a ket, producing a number
\[
\bra{A} ≃ ∫ \ \mathrm{d}x A^{*}(x)
\]
\[
\bra{A}\ket{B} = ∫ \ \mathrm{d}x A^{*}(x)B(x)
\]
\[
\bra{B}\ket{A} = ∫ \ \mathrm{d}x B^{*}(x) A(x)
\]

\section{Sets of Kets}
this is a ket $|u⟩$ 

\section{Discrete and Continuous Basis} 
\subsection{Discrete}
\[
\ket{f} = ∑_{i=1}^{∞} f_i \ket{i}
\]
For orthonormal basis 
\[
\bra{i}\ket{j} = δ_{ij} → f_j = \bra{j}\ket{f}
\]

\subsection{Continuous}
\[
\ket{f} = ∫_{0}^{L}  f(x')\ket{x'} \ \mathrm{d}x
\]
\[
\bra{x}\ket{f} = f(x)
\]

\[
\bra{x}\ket{f} = ∫_{0}^{L} \bra{x} f(x) \ket{x'} \ \mathrm{d}x = ∫_{0}^{L} f(x) \underbrace{\bra{x}\ket{x'}}_{δ} \ \mathrm{d}x = f(x)
\]

In a short interval $[-ϵ, ϵ]$ the function $f$ becomes approximately constant. Using the definition of the Dirac-delta we get the following
\[
\lim_{ϵ \to 0} ∫_{-ϵ}^{ϵ} f(x) δ(x-x') \ \mathrm{d}x = f(x) \ \underbrace{ \lim_{ϵ \to 0} ∫_{-ϵ}^{ϵ} δ(x-x') \ \mathrm{d}x}_{1}=  f(x)
\]


\end{document}