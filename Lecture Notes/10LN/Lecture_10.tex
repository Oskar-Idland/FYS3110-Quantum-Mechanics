\documentclass{article}
\usepackage{amsmath}
\usepackage[mathletters]{ucs}
\usepackage[utf8x]{inputenc}
\usepackage[margin=1.5in]{geometry}
\usepackage{enumerate}
\newtheorem{theorem}{Theorem}
\usepackage[dvipsnames]{xcolor}
\usepackage{pgfplots}
\setlength{\parindent}{0cm}
\usepackage{graphics}
\usepackage{graphicx} % Required for including images
\usepackage{subcaption}
\usepackage{bigintcalc}
\usepackage{pythonhighlight} %for pythonkode \begin{python}   \end{python}
\usepackage{appendix}
\usepackage{arydshln}
\usepackage{physics}
\usepackage{booktabs} 
\usepackage{adjustbox}
\usepackage{mdframed}
\usepackage{relsize}
\usepackage{physics}
\usepackage[thinc]{esdiff}
\usepackage{fixltx2e}
\usepackage{esint}  %for lukket-linje-integral
\usepackage{xfrac} %for sfrac
\usepackage[colorlinks=true]{hyperref} %for linker, må ha med hypersetup
\usepackage[noabbrev, nameinlink]{cleveref} % to be loaded after hyperref
\usepackage{amssymb} %\mathbb{R} for reelle tall, \mathcal{B} for "matte"-font
\usepackage{listings} %for kode/lstlisting
\usepackage{verbatim}
\usepackage{graphicx,wrapfig,lipsum,caption} %for wrapping av bilder
\usepackage{mathtools} %for \abs{x}
\usepackage[norsk]{babel}
\definecolor{codegreen}{rgb}{0,0.6,0}
\definecolor{codegray}{rgb}{0.5,0.5,0.5}
\definecolor{codepurple}{rgb}{0.58,0,0.82}
\definecolor{backcolour}{rgb}{0.95,0.95,0.92}
\lstdefinestyle{mystyle}{
    backgroundcolor=\color{backcolour},   
    commentstyle=\color{codegreen},
    keywordstyle=\color{magenta},
    numberstyle=\tiny\color{codegray},
    stringstyle=\color{codepurple},
    basicstyle=\ttfamily\footnotesize,
    breakatwhitespace=false,         
    breaklines=true,                 
    captionpos=b,                    
    keepspaces=true,                 
    numbers=left,                    
    numbersep=5pt,                  
    showspaces=false,                
    showstringspaces=false,
    showtabs=false,                  
    tabsize=2
}

\lstset{style=mystyle}
\author{Oskar Idland}
\title{Lecture 10}
\date{}
\begin{document}
\maketitle
\newpage
\section*{Spin 1/2}
\[
\hat{S} = (S_x, S_y, S_z)
\] 
An internal degree of freedom, which means it has nothing to do with orientation in space. 
\[
[S_y, S_x] = iℏS_z
\]
\[
[S_z, S_y] = iℏS_x
\]
\[
[S_x, S_z] = iℏS_y
\]
\[
S^2 = S_x^2 + S_y^2 + S_z^2
\]
\[
S^2 \ket{s,m} = ℏ^2 s(s+1) \ket{s,m}
\]
where $s$ is the quantum number, where $s ∈ \left\{0, 1 / 2, 1, \ldots  \right\}$ and $m ∈ \left\{-s, -s +1, \ldots  , s\right\}$
\[
S_{\pm} = S_x \pm iS_y
\]
\[
S_{\pm} \ket{s,m} = ℏ \sqrt{s(s+1) - m(m \pm 1)} \ket{s, m \pm 1}
\]
When the spin is 1/2, we have two states, $\ket{1/2, +1/2}$ and $\ket{1/2, -1/2}$. This can also be written as $\ket{↑_z}$ and $\ket{↓_z}$ or $\ket{1}$ and $\ket{0}$.
\subsection*{Applying the spin operators}
\[
S_z \ket{↑_z} = \frac{ℏ}{2} \ket{↑_z}
\]
\[
S_z \ket{↓_z} = -\frac{ℏ}{2} \ket{↓_z}
\]
The state are naturally orthonormal, meaning $\bra{↑}\ket{↓} = 0$ and $\bra{↑}\ket{↑} = 1$ and $\bra{↓}\ket{↓} = 1$.  
\subsection*{Rewriting the operator in bracket notation in the z-basis}
\[
S_z = \frac{ℏ}{2} \Big(\ket{↑}\bra{↑} - \ket{↓}\bra{↓}\Big)
\]
\subsubsection*{Lowering and raising operators}
\[
S_+ = ℏ \ket{↑}\bra{↓}
\]
\[
S_- = ℏ \ket{↓}\bra{↑}
\]
\[
S_{+}\ket{↓} = ℏ\ket{↑} \quad , \quad S_{-}\ket{↓} = 0
\]

\subsection*{2D representation of the operators}
\[
\ket{↑} ≃ \begin{pmatrix}
1\\
0
\end{pmatrix} \quad , \quad \ket{↓} ≃ 
\begin{pmatrix}
0\\
1
\end{pmatrix}
\]
\subsubsection*{State vectors}
\[
\ket{↑}\bra{↑} ≃ 
\begin{pmatrix*}[r]
 1 & 0 \\
 0 & 0 \\
\end{pmatrix*} \quad , \quad \ket{↓}\bra{↓} ≃
\begin{pmatrix*}[r]
 0 & 0 \\
 0 & 1 \\
\end{pmatrix*}
\]
\subsubsection*{Spin operators}
\[
S_z = ℏ / 2 \left(\begin{pmatrix*}[r]
 1 & 0 \\
 0 & 0 \\
\end{pmatrix*} -
\begin{pmatrix*}[r]
 0 & 0 \\
 0 & 1 \\
\end{pmatrix*}\right) = ℏ / 2 \begin{pmatrix*}[r]
 1 & 0 \\
 0 & -1 \\
\end{pmatrix*}
\]
\[
S_x = 
ℏ / 2 \begin{pmatrix*}[r]
 0 & 1 \\
 1 & 0 \\
\end{pmatrix*}
\]
\[
S_y = ℏ / 2
\begin{pmatrix*}[r]
 0 & -i \\
 i & 0 \\
\end{pmatrix*}
\]
The matrixes are often written as $σ_z, σ_x, σ_y$. 
\[
S^2 = S_x^2 + S_y^2 + S_z^2 = \frac{3ℏ^2}{4} \begin{pmatrix*}[r]
 1 & 0 \\
 0 & 1 \\
\end{pmatrix*}
\]

\section*{Rotating charge in a magnetic field}
\subsection*{Classical}
\[
I = \frac{1}{T} = \frac{q}{2πr} = \frac{qv}{2πr}
\]
A circular current is like a small magnet, where $I$ is the current, $T$ is the period, $q$ is the charge, $r$ is the radius and $v$ is the velocity. The magnetic moment is defined as $\vec{µ} = I\vec{A} = \frac{qvr}{2}$, where $\vec{A}$ is the area vector.
\subsubsection*{Angular momentum}
\[
\left|\vec{L}\right| = mvr \quad , \quad \left|\vec{μ}\right| = \frac{2}{2m}\hat{L}
\]
where $m$ is the mass of the particle.
\[
H = -\vec{µ} ⋅ \vec{B} 
\]

\subsubsection*{Electron spin in a magnetic field}
\[
\vec{μ} = γ \vec{S} 
\]
where $γ$ is the gyromagnetic ratio $ = 2.0023 \left( - e /2m\right)$
\[
H = - γ \vec{S} ⋅ \vec{B}
\]

\subsubsection*{Spin motion in a magnetic field}
\paragraph{Generic State: }
\[
\ket{ψ} = a\ket{↑} + b\ket{↓}
\]
Parameterize the coefficients
\[
a = \cos \frac{α}{2} e^{iϕ} \quad , \quad b = \sin \frac{α}{2} e^{iφ}
\]
put the <-axis along $\vec{B}$. 
\[
H = - γ \vec{S} ⋅ \vec{B} = - γ B S_z
\]
The eigenstates of the Hamiltonian $H$ is $\ket{↑}$ and $\ket{↓}$, with eigenvalues $-γB ℏ / 2$ and $γB ℏ / 2$ respectively. We can then rewrite the generic state as follows:
\[
\ket{ψ(t)} = e^{-iHt / ℏ} \ket{ψ} = e^{-iHt / ℏ} \left(a\ket{↑} + b \ket{↓}\right)
\]
\[
\ket{ψ(t)} = e^{iγBt / 2} a\ket{↑} + e^{-iγBt / 2} b\ket{↓}
\]

\paragraph{Applying the $S_i$ operator}
\[
\bra{ψ(t)}S_z\ket{ψ(t)} = \left( a^{*} e^{-iγBt / 2} \bra{↑} + b^{*} e^{iγBt / 2} \bra{↓}\right) \left(\ket{↑}\bra{↑} - \ket{↓}\bra{↓}\right) \left(e^{iγBt / 2} a \ket{↑} + e^{-iγBt / 2}b \ket{↓}\right)
\]
\[
\left<S_z\right>= ℏ/2 \left(a^{*}a - b^{*} b\right) = \frac{ℏ}{2} \cos α
\]
The result is time independent. 
\paragraph{Expectation value of $S_x$}
\[
\left<S_x\right> = \left( a^{*} e^{-iγBt / 2} \bra{↑} + b^{*} e^{iγBt / 2} \bra{↓}\right) ℏ / 2\left(\ket{↓}\bra{↑} + \ket{↑}\bra{↓}\right) \left(e^{iγBt / 2} a \ket{↑} + e^{-iγBt / 2}b \ket{↓}\right)
\]
\[
\left<S_x\right> = \frac{ℏ}{2} \left(a^{*}b e^{-iγBt} + a b^{*} e^{iγBt}\right)
\]
\[
\left<S_x\right> = \frac{ℏ}{2} \left( be^{-iγBt / 2} \ket{↑} + ae ^{iγBt / 2}\ket{↓}\right)
\]
\[
\left<S_x\right> = \frac{ℏ}{2} \cos (α / 2) \sin (α / 2) e^{-iϕ} e^{iθ} e^{-iγBt} + \underbrace{c.c}_{\text{complex conjugate}}
\]
\[
\left<S_x\right> = \frac{ℏ}{2} \sin α \cos (θ - ϕ - γBt)
\]

The result is time dependent and oscillates just like you would expect classically. 
\[
\left<S_y\right> = - \frac{ℏ}{2}\sin α\cos (γBt + ϕ - θ) 
\]
The expectation value of the spin precess around the field direction, with a frequency of $w_{L} = γB$

\section*{Composite systems}
Now we combine the degrees of freedom coming. 
\subsection*{A particle with real space state $\ket{nlm}$ and spin $\ket{↑}$}
\[
\ket{nlm↑} = \ket{nlm} ⊗ \ket{↑} 
\]
\[
\ket{nlm} = ψ_{nlm}(γ,θ,ϕ)
\]
\[
\ket{s} = a \ket{↑} + b \ket{↓} = 
\begin{pmatrix*}[r]
 a \\
 b \\
\end{pmatrix*}
\]
The tensor product becomes a combination of all possible states:
\[
\ket{nlm,s} = \ket{nlm} ⊗ \ket{s} =
\begin{pmatrix*}[r]
 a ψ_{nlm}\ket{nlm} \\
 b ψ_{nlm}\ket{nlm} \\
\end{pmatrix*}
\]
\subsection*{Two spin 1/2 particles}
\[
\ket{a} = 
\begin{pmatrix*}[r]
 a_1 \\
 a_2 \\
\end{pmatrix*}
\quad , \quad
\ket{b} =
\begin{pmatrix*}[r]
 b_1 \\
 b_2 \\
\end{pmatrix*}
\]
\[
\ket{a} ⊗ \ket{b} =
\begin{pmatrix*}[r]
 a_1 b_1 \\
 a_1 b_2 \\
 a_2 b_1 \\
 a_2 b_2 \\
\end{pmatrix*}
\]
With dimensions $4 ×1 $. 
\subsection*{Properties of Tensor Products}
\[
\ket{ϕ} = ∑_{}^{} c_n \ket{n} \quad , \quad \ket{x} = ∑_{x| }^{} d_m \ket{m}
\]
\begin{itemize}
    
    \item The dimensions of the tensor product is the product of the dimensions of the two vectors. $a×b ⊗ c×d = ac×bd$
    \item 
    \[
    \ket{ϕ} ⊗ \ket{x} = ∑_{}^{} c_n d_m \ket{n} ⊗ \ket{m} = α \ket{ϕ} ⊗ \ket{x_1} + β \ket{ϕ} ⊗ \ket{x_2}
    \]
    \item
    \[
    \left(\bra{n'} \otimes \bra{m'}\right) \left(\ket{n} \otimes \ket{m}\right) = \bra{n'}\ket{n} ⋅ \bra{m'}\ket{m}
    \]

\end{itemize}
\subsection*{Measurement with more degrees of freedom}


\end{document}