\documentclass{article}
\usepackage{amsmath}
\usepackage[mathletters]{ucs}
\usepackage[utf8x]{inputenc}
\usepackage[margin=1.5in]{geometry}
\usepackage{enumerate}
\newtheorem{theorem}{Theorem}
\usepackage[dvipsnames]{xcolor}
\usepackage{pgfplots}
\setlength{\parindent}{0cm}
\usepackage{graphics}
\usepackage{graphicx} % Required for including images
\usepackage{subcaption}
\usepackage{bigintcalc}
\usepackage{pythonhighlight} %for pythonkode \begin{python}   \end{python}
\usepackage{appendix}
\usepackage{arydshln}
\usepackage{physics}
\usepackage{booktabs} 
\usepackage{adjustbox}
\usepackage{mdframed}
\usepackage{relsize}
\usepackage{physics}
\usepackage[thinc]{esdiff}
\usepackage{fixltx2e}
\usepackage{esint}  %for lukket-linje-integral
\usepackage{xfrac} %for sfrac
\usepackage{hyperref} %for linker, må ha med hypersetup
\usepackage[noabbrev, nameinlink]{cleveref} % to be loaded after hyperref
\usepackage{amssymb} %\mathbb{R} for reelle tall, \mathcal{B} for "matte"-font
\usepackage{listings} %for kode/lstlisting
\usepackage{verbatim}
\usepackage{graphicx,wrapfig,lipsum,caption} %for wrapping av bilder
\usepackage{mathtools} %for \abs{x}
\usepackage[norsk]{babel}
\usepackage{cancel}
\definecolor{codegreen}{rgb}{0,0.6,0}
\definecolor{codegray}{rgb}{0.5,0.5,0.5}
\definecolor{codepurple}{rgb}{0.58,0,0.82}
\definecolor{backcolour}{rgb}{0.95,0.95,0.92}
\lstdefinestyle{mystyle}{
    backgroundcolor=\color{backcolour},   
    commentstyle=\color{codegreen},
    keywordstyle=\color{magenta},
    numberstyle=\tiny\color{codegray},
    stringstyle=\color{codepurple},
    basicstyle=\ttfamily\footnotesize,
    breakatwhitespace=false,         
    breaklines=true,                 
    captionpos=b,                    
    keepspaces=true,                 
    numbers=left,                    
    numbersep=5pt,                  
    showspaces=false,                
    showstringspaces=false,
    showtabs=false,                  
    tabsize=2
}

\lstset{style=mystyle}
\author{Oskar Idland}
\title{Lecture Notes 3}
\date{}
\begin{document}
\maketitle
\newpage
\section*{Operators}
\subsection*{Hermitian Conjugate}
\paragraph{Definition:} 
\[
\bra{v}\hat{K}^{†}\ket{u} = \bra{u}\hat{K}\ket{v}^{*}
\]
\subsubsection*{Discrete basis}
\[
\underbrace{\bra{n}\hat{K}^{†}\ket{m}}_{K^{†}_{nm}} =
\underbrace{\bra{m}\hat{K}\ket{n}^{*}}_{K_{mn}^{*}}
\]
\[
K^{†} = K^{*T} = K → K^{†}_{nm} = K_{mn}^{*} = K_{nm} → K_{nn} ∈ ℝ
\]
\[
\text{when } n ≠ m: K_{nm} = K_{mn}^{*} = K_{nm}^{†}
\]

\subsection*{Spectrum of an Operator}
\paragraph{Definition:} The spectrum of an operator $\hat{K}$ is the set of all eigenvalues of $\hat{K}$.
Two or more linearly independent eigenvectors $\ket{λ_i}$ have the same eigenvalue $λ$, the spectrum is said to be degenerate. We can always choose the eigenvectors to be orthonormal. If there are $g$ states $\ket{λ_i}$ with eigenvalue $λ$, then the level degeneracy is $g$.

\subsection*{Hermitian Operators}
\paragraph{Properties}
\begin{itemize}
    \item Eigenvalues are real
    \item Different eigenvalues correspond to orthogonal eigenvectors
    \item Eigenvectors with the same eigenvalues can be chosen to be orthogonal
    \item The eigenkets from a complete set of basis vectors for a finite dimensional Hilbert space. 
\end{itemize}

\paragraph{Proof of Eigenvectors Creating a Liner Compination which is also an Eigenvector}
\[
\hat{K}α \ket{λ_1} = λ_1 α \ket{λ_1} \quad \hat{K}β \ket{λ_2} = λ_2 β \ket{λ_2}
\]
\[
\hat{K}\left(α\ket{λ_1} + β\ket{λ_2}\right) = λ (α\ket{λ_1} + β\ket{λ_2})
\]

\subsection*{Spectral Representation of Operators}
\paragraph{Definition:} The spectral representation of an operator $\hat{K}$ in its basis of its eigenkets. 
\[
\bra{λ_i}\hat{K}\ket{λ_j} = \bra{λ_i}\ket{λ_j}λ_j = δ_{ij}λ_j
\]

This shows that the matrix elements of a Hermitian operator in its eigenket basis are on the diagonal.
\[
\hat{K} ≃ 
\begin{bmatrix*}[r]
 λ_1 & 0 & 0 \\
 0 & λ_2 & 0 \\
 0 & 0 & λ_3 
\end{bmatrix*}
\]
We guess that (this is the spectral representation)
\[
\hat{K} = ∑_{r}^{} λ_r \ket{λ_r}\bra{λ_r} 
\]
\[
\bra{λ_i}\hat{K}\ket{λ_j} = \bra{λ_i}∑_{r}^{} \ket{λ_r}\bra{λ_r}\ket{λ_j} = ∑_{r}^{} λ_r \bra{λ_i}\ket{λ_j} \bra{λ_r}\ket{λ_j}
\]
$r$ must be equal to both $i$ and $j$ for the sum to be non-zero.
\[
λ_j δ_{ij}
\]
\subsection*{Physical Meaning}
\paragraph{Eigenvalues: } Measurement value $λ$
\paragraph{Eigenket: } State on which a measurement of the quantity represented by $\hat{K}$, gives the value $λ$ with certainty.




\end{document}