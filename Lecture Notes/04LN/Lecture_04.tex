\documentclass{article}
\usepackage{amsmath}
\usepackage[mathletters]{ucs}
\usepackage[utf8x]{inputenc}
\usepackage[margin=1.5in]{geometry}
\usepackage{enumerate}
\newtheorem{theorem}{Theorem}
\usepackage[dvipsnames]{xcolor}
\usepackage{pgfplots}
\pgfplotsset{compat=1.18}
\setlength{\parindent}{0cm}
\usepackage{graphics}
\usepackage{graphicx} % Required for including images
\usepackage{subcaption}
\usepackage{bigintcalc}
\usepackage{pythonhighlight} %for pythonkode \begin{python}   \end{python}
\usepackage{appendix}
\usepackage{arydshln}
\usepackage{physics}
\usepackage{booktabs} 
\usepackage{adjustbox}
\usepackage{mdframed}
\usepackage{relsize}
\usepackage{physics}
\usepackage[thinc]{esdiff}
\usepackage{esint}  %for lukket-linje-integral
\usepackage{xfrac} %for sfrac
\usepackage{hyperref} %for linker, må ha med hypersetup
\usepackage[noabbrev, nameinlink]{cleveref} % to be loaded after hyperref
\usepackage{amssymb} %\mathbb{R} for reelle tall, \mathcal{B} for "matte"-font
\usepackage{listings} %for kode/lstlisting
\usepackage{verbatim}
\usepackage{graphicx,wrapfig,lipsum,caption} %for wrapping av bilder
\usepackage{mathtools} %for \abs{x}
\usepackage[norsk]{babel}
\usepackage{cancel}
\definecolor{codegreen}{rgb}{0,0.6,0}
\definecolor{codegray}{rgb}{0.5,0.5,0.5}
\definecolor{codepurple}{rgb}{0.58,0,0.82}
\definecolor{backcolour}{rgb}{0.95,0.95,0.92}
\lstdefinestyle{mystyle}{
    backgroundcolor=\color{backcolour},   
    commentstyle=\color{codegreen},
    keywordstyle=\color{magenta},
    numberstyle=\tiny\color{codegray},
    stringstyle=\color{codepurple},
    basicstyle=\ttfamily\footnotesize,
    breakatwhitespace=false,         
    breaklines=true,                 
    captionpos=b,                    
    keepspaces=true,                 
    numbers=left,                    
    numbersep=5pt,                  
    showspaces=false,                
    showstringspaces=false,
    showtabs=false,                  
    tabsize=2
}

\lstset{style=mystyle}
\author{Oskar Idland}
\title{Lecture 4}
\date{}
\begin{document}
\maketitle
\newpage

\section*{Postulates of Quantum Mechanics}
\subsection*{1. The ket representing the state is normalizable}
$\ket{ϕ}$ and $e^{iθ}\ket{ϕ}$ is the same state. 

\subsection*{2. Observable quantities are represented by Hermitian operators}
\paragraph{Erhenfest Theorem: }
Expectation values of measurements obey classical eq. of motion. 

\subsection*{3. Measurement values of $K$ is the eigenvalues of $\hat{K}$}
\[
\left<K\right>_{ψ} = ∑_{i=1}^{D} λ_{i} P_{λ_{i}} = ∑_{i=1}^{D } λ_{i} {\bra{ψ}\ket{λ_{i}}} \bra{λ_{i}}\ket{ψ}
\]
\[
\bra{ψ} ∑_{i=1}^{D} \ket{λ_{i}} \bra{λ_{i}} λ_i = \bra{ψ} \hat{K} \ket{ψ} 
\]


\subsection*{4. The probability of getting value* $λ$ when measuring $K$ in state $\ket{ψ}$ is $\bra{ψ}\hat{P}_{λ}\ket{ψ}$ where $\hat{P}_{λ} = \ket{λ}\bra{λ}$}
Degenerate states:
\[
\bra{ψ}P_{λ_{i}}\ket{ψ} = ∑_{j=1}^{g} \bra{ψ}\ket{λ_{i}^{j}} \bra{λ_{i}^{j}}\ket{ψ} 
\]
where $g$ is the degree of degeneracy. 

Continuous eigenvalues $λ$: 
\[
P(λ, λ + Δλ) = ∫_{λ}^{λ + Δλ} \bra{ψ}\underbrace{\ket{λ'} \bra{λ'}}_{\hat{P}_x}\ket{ψ} \ \mathrm{d}λ'
\]


\subsection*{5. State after an ideal measurement which gave $λ$ is $\propto \hat{P}_{λ}\ket{pdi}$}

\subsection*{6. Time evolution: $iℏ \frac{\mathrm{d}}{\mathrm{d}t} \ket{ψ(t)} = \hat{H} \ket{ψ(t)}$}


\section*{Interpretation}
\subsection*{Minimal approach (where everyone agrees)}
\begin{itemize}
    \item QM makes statements about observable quantities
    \item Probabilities corresponds to relative number of occurrences in repeated experiments. 
\end{itemize}

\subsection*{Going Further}
\subsubsection*{Instrumentalist}    
\begin{itemize}
    \item Inappropriate to consider an individual system to have definite values for all its physical properties. 
    \item Measurement is fundamental and QM is only concerned with the outcome of measurements.
\end{itemize}

\subsubsection*{Realist}
\begin{itemize}
    \item An individual system has definite values for all its physical properties.
    \item Measurement is not fundamental.
    \item Probabilities reflect our ignorance of the system.
    \item QM is incomplete. There need to be an underlying theory explaining the hidden variables
\end{itemize}
John Bell: Created inequalities that must be be satisfied for such a theory, but are violated by QM.


\section*{Compatible Operators}
Two operators are compatible if they share a common complete set of eigenkets (not necessarily eigenvalues). They will commute. If they do not commute, they are not compatible. 
\[
\hat{A} \ket{ψ_n} = a_n \ket{ψ_n}
\]
\[
\hat{B} \ket{ψ_n} = b_n \ket{ψ_n}
\]
\[
\left\{\ket{ψ_n}\right\} \text{ spans the Hilbert space}
\]

\[
\hat{A}\hat{B} \ket{ψ_n} - \hat{B}\hat{A} \ket{ψ_n} = \hat{A} b_n \ket{ψ_n} - \hat{B} a_n \ket{ψ_n} = (a_n b_n - b_n a_n) \ket{ψ_n} = 0
\]


\end{document}