\documentclass{article}
\usepackage{amsmath}
\usepackage[mathletters]{ucs}
\usepackage[utf8x]{inputenc}
\usepackage[margin=1.5in]{geometry}
\usepackage{enumerate}
\newtheorem{theorem}{Theorem}
\usepackage[dvipsnames]{xcolor}
\usepackage{pgfplots}
\pgfplotsset{compat=1.18}
\setlength{\parindent}{0cm}
\usepackage{graphics}
\usepackage{graphicx} % Required for including images
\usepackage{subcaption}
\usepackage{bigintcalc}
\usepackage{pythonhighlight} %for pythonkode \begin{python}   \end{python}
\usepackage{appendix}
\usepackage{arydshln}
\usepackage{physics}
\usepackage{booktabs} 
\usepackage{adjustbox}
\usepackage{mdframed}
\usepackage{relsize}
\usepackage{physics}
\usepackage[thinc]{esdiff}
\usepackage{esint}  %for lukket-linje-integral
\usepackage{xfrac} %for sfrac
\usepackage{hyperref} %for linker, må ha med hypersetup
\usepackage[noabbrev, nameinlink]{cleveref} % to be loaded after hyperref
\usepackage{amssymb} %\mathbb{R} for reelle tall, \mathcal{B} for "matte"-font
\usepackage{listings} %for kode/lstlisting
\usepackage{verbatim}
\usepackage{graphicx,wrapfig,lipsum,caption} %for wrapping av bilder
\usepackage{mathtools} %for \abs{x}
\usepackage[norsk]{babel}
\usepackage{cancel}
\definecolor{codegreen}{rgb}{0,0.6,0}
\definecolor{codegray}{rgb}{0.5,0.5,0.5}
\definecolor{codepurple}{rgb}{0.58,0,0.82}
\definecolor{backcolour}{rgb}{0.95,0.95,0.92}
\lstdefinestyle{mystyle}{
    backgroundcolor=\color{backcolour},   
    commentstyle=\color{codegreen},
    keywordstyle=\color{magenta},
    numberstyle=\tiny\color{codegray},
    stringstyle=\color{codepurple},
    basicstyle=\ttfamily\footnotesize,
    breakatwhitespace=false,         
    breaklines=true,                 
    captionpos=b,                    
    keepspaces=true,                 
    numbers=left,                    
    numbersep=5pt,                  
    showspaces=false,                
    showstringspaces=false,
    showtabs=false,                  
    tabsize=2
}

\lstset{style=mystyle}
\author{Oskar Idland}
\title{Lecture 13}
\date{}
\begin{document}
\maketitle
\newpage
\part*{Indistinguishable Particles: Bosons and Fermions}
\section*{Two-particle state}
A sum of two particle states is the tensor product of the two states:
\[
\ket{ψ} = \ket{a} ⊗ \ket{b} + \ket{c} ⊗ \ket{d}
\]
where $\ket{ψ}$ is a superposition of two two-particle states. We have given them labels, but that is unnecessary if the particles are indistinguishable.
\paragraph{Prompt: }
If the particles are indistinguishable, there should be symmetry under permutation of the particles.
\section*{Permutation Operator}
\[
\hat{P} \ket{a} ⊗ \ket{b} = \ket{b} ⊗ \ket{a}
\]
Finding the eigenvalues of $\hat{P}$:
\[
\hat{P}^2 \ket{a} ⊗ \ket{b} = \hat{P} \ket{b} ⊗ \ket{a} = \ket{a} ⊗ \ket{b}
\]
\[
\hat{P}^2 = \hat{I} → \hat{P} = \hat{P}^{-1}
\]
Checking if the operator is Hermitian:
\[
\bra{c} ⊗ \bra{d} \hat{P}^{†} \ket{a} ⊗ \ket{b} = \bra{c} ⊗ \bra{d} \hat{P} \ket{a} ⊗ \ket{b}^{*} 
\]
Looking at the right hand side:
\[
\bra{a} ⊗ \bra{b}\hat{P} \ket{c} ⊗ \ket{d}^{*} = \bra{a} ⊗ \bra{b} \ket{d} ⊗ \ket{c}^{*} 
\]
\[
\bra{a}\ket{d}^{*} \bra{b} \ket{c}^{*} = \bra{d}\ket{a}\bra{c}\ket{b}
\]
\[
\bra{c} ⊗ \bra{d} \hat{P} \ket{a} ⊗ \ket{b} = \bra{c}\ket{b}\bra{d}\ket{a} 
\]
which is the same as the previous expression.

The last equation tells us that $\hat{P}^{†} = \hat{P}$. The eigenvalues of $\hat{P}$ are real with eigenvalues of $\pm 1$. 
\paragraph{Conclusion: }
\[
\hat{P} \ket{ψ_+} = ± \ket{ψ_±}
\] 
where the plus sign is for bosons and the minus sign is for fermions.

\paragraph{Fact: }
For identical particles, $\left[\hat{H}, \hat{P}\right] = 0$. Eigenstates of $\hat{H}$are  also eigenkets of $\hat{P}$. Identical particles are only one type $λ = \pm 1$. Plus for symmetric and minus for antisymmetric.

\paragraph{Fact: }
In $\dim \le 2$, $λ = e^{iθ}$ Anyons are neither bosons nor fermions. They are found in 2D/1D systems. This was discovered by not setting labels on particles.  

\section*{Example: 2-particle state}
\[
\ket{ψ} = \frac{1}{\sqrt{2}} \left(\ket{a} ⊗ \ket{b} \overbrace{\underbrace{±}_{\text{Fermion}}}^{\text{Boson}} \ket{b} ⊗ \ket{a}\right)
\]
\[
\ket{ψ_3} = \frac{1}{\sqrt{3}} \left(\ket{a} ⊗ \ket{b} ⊗ \ket{c} - \ket{b} ⊗ \ket{a} ⊗ \ket{c} + \ket{b} ⊗ \ket{c} ⊗ \ket{a} - \ket{a} ⊗ \ket{c} ⊗ \ket{b} + \ket{c} ⊗ \ket{a} ⊗ \ket{b} - \ket{c} ⊗ \ket{b} ⊗ \ket{a}\right) 
\]
This state is antisymmetric under $P_{12}, P_{13}, P_{23}$. 
\subsection*{Coordinate representation}
\[
\left(\bra{r_1} ⊗ \bra{r_2}\right) \ket{ψ_2} = \frac{1}{\sqrt{2}} \left(\bra{r_1}\ket{a} \bra{r_2}\ket{b} \pm \bra{r_1}\ket{b} \bra{r_2}\ket{a}\right)
\]
\[
\frac{1}{\sqrt{2}} 
\begin{vmatrix*}[r]
 ψ_a(r_1) & ψ_a(r_2) \\
 ψ_b(r_1) & ψ_b(r_2) \\
\end{vmatrix*}_{\pm}
\]
\paragraph{Warning: }
The fact that half-spin particles are fermions and integer spin-particles are bosons is a consequence of having a relativistic quantum field theory. It does not hold generally. 

\section*{Pauli Principle}
\[
\ket{ψ_2} = \frac{1}{\sqrt{2}} \left(\ket{a} ⊗ \ket{b} - \ket{b} ⊗ \ket{a}\right)
\]
If $a = b$ then $\ket{ψ_2} = \ket{\text{NULL}}$ which is an unphysical state. Two Fermions can't have the same single-particle state (quantum numbers). 

\section*{Exchange Forces}
Study of the separation between two particles in 1D. 
\[
\hat{D} = \hat{X} ⊗ \hat{I} - \hat{I} ⊗ \hat{X}
\]
Consider $\hat{D}^2$
\[
\hat{D}^2 = (X ⊗ I - I ⊗ X) (X ⊗ I - I ⊗ X)
\]
\[
\hat{D}^2 = X^2 ⊗ I+  I ⊗ X^2 - 2 X ⊗ X
\]
Distinguishable particles: 
\[
\ket{ψ_0} = \ket{a} ⊗ \ket{b}
\]
Identical particles:
\[
\ket{ψ_{\pm}} = \frac{1}{\sqrt{2}} \left(\ket{a} ⊗ \ket{b} ± \ket{b} ⊗ \ket{a}\right)
\]
\subsection*{Distinguishable particles}
\[
\bra{ψ_D}\hat{D}^2\ket{ψ_D} = \bra{a} ⊗ \bra{b} \hat{D}^2 \ket{a} ⊗ \ket{b}
\]
\[
\bra{a} ⊗ \bra{b} \left(X^2 \ket{a} ⊗ \ket{b} + \ket{a} ⊗ X^2\ket{b} - 2X \ket{a} ⊗ X\ket{b}\right)
\]
\[
\bra{a}X^2\ket{a} + \bra{b}X^2\ket{b} - 2\bra{a}X\ket{a}\bra{b}X\ket{b}
\]

\subsection*{Identical particles}
\[
\bra{Ψ_\pm}\hat{D}^2\ket{ψ_\pm} = \frac{1}{2} \left(\bra{a} ⊗ \bra{b} \pm \bra{b} ⊗ \bra{a}\right) \left(X^2 \ket{a} ⊗ \ket{b} + \ket{a} ⊗ X^2\ket{b} - 2X \ket{a} ⊗ X\ket{b}\right) ± \left(\hat{X}\ket{b} ⊗ \ket{a} + \ket{b} ⊗ X^2\ket{a} - 2X \ket{b} ⊗ X\ket{a}\right)
\]

Calculating the inner product:
\[
\bra{Ψ_\pm}\hat{D}^2\ket{ψ_\pm} = \frac{1}{2} \left(\bra{a}X^2\ket{a} + \bra{b}X^2\ket{b} - 2 \bra{a}X\ket{a} \bra{b}X\ket{b}\right)
\]
\[
\bra{Ψ_\pm}\hat{D}^2\ket{ψ_\pm} = ± \left(-2 \bra{a}X\ket{b} \bra{b}X\ket{a}\right)± \left(-2 \bra{b}X\ket{a} \bra{a}X\ket{b} ± \bra{b}X^2\ket{b} + \bra{a}X^2\ket{a} - 2 \bra{b}X\ket{b} \bra{a}X\ket{a}\right) 
\]
\[
\bra{Ψ_\pm}\hat{D}^2\ket{ψ_\pm} = \underbrace{\bra{a}X^2\ket{a} + \bra{b}X^2\ket{b} -2  \bra{a}X\ket{a} \bra{b}x\ket{b}}_{\bra{ψ_0}\hat{D}^2\ket{ψ_0}} ∓ 2 \bra{a}X\ket{b} \underbrace{\bra{b}X\ket{a}}_{\bra{a}X\ket{b}^{*}} 
\]
\[
\bra{ψ_\pm}\hat{D}^2\ket{ψ_\pm}=  \bra{ψ_0}\hat{D}^2\ket{ψ_0} ∓ 2 \left|\bra{a}X\ket{b}\right|^2
\]
The minus belongs to the symmetrical combinations, and the plus belongs to the antisymmetrical combinations. 
\paragraph{Conclusion: }
Separation between particles in an antisymmetrical state is generally larger than the separation between particles in a symmetrical state. The effect of exchange forces depends on how close the states are together. We see this in the last term, where we see how much the wavefunctions overlap. No overlap, means we don't have to worry about exchange forces.

\paragraph{Conclusion: }
When the orbitals of two electrons are closer together, they tend to be in an symmetrical state. This is because the symmetrical state has a smaller distance between the particles.


\end{document}