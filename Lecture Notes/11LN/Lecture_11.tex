\documentclass{article}
\usepackage{amsmath}
\usepackage[mathletters]{ucs}
\usepackage[utf8x]{inputenc}
\usepackage[margin=1.5in]{geometry}
\usepackage{enumerate}
\newtheorem{theorem}{Theorem}
\usepackage[dvipsnames]{xcolor}
\usepackage{pgfplots}
\pgfplotsset{compat=1.18}
\setlength{\parindent}{0cm}
\usepackage{graphics}
\usepackage{graphicx} % Required for including images
\usepackage{subcaption}
\usepackage{bigintcalc}
\usepackage{pythonhighlight} %for pythonkode \begin{python}   \end{python}
\usepackage{appendix}
\usepackage{arydshln}
\usepackage{physics}
\usepackage{booktabs} 
\usepackage{adjustbox}
\usepackage{mdframed}
\usepackage{relsize}
\usepackage{physics}
\usepackage[thinc]{esdiff}
\usepackage{esint}  %for lukket-linje-integral
\usepackage{xfrac} %for sfrac
\usepackage{hyperref} %for linker, må ha med hypersetup
\usepackage[noabbrev, nameinlink]{cleveref} % to be loaded after hyperref
\usepackage{amssymb} %\mathbb{R} for reelle tall, \mathcal{B} for "matte"-font
\usepackage{listings} %for kode/lstlisting
\usepackage{verbatim}
\usepackage{graphicx,wrapfig,lipsum,caption} %for wrapping av bilder
\usepackage{mathtools} %for \abs{x}
\usepackage[norsk]{babel}
\usepackage{cancel}
\definecolor{codegreen}{rgb}{0,0.6,0}
\definecolor{codegray}{rgb}{0.5,0.5,0.5}
\definecolor{codepurple}{rgb}{0.58,0,0.82}
\definecolor{backcolour}{rgb}{0.95,0.95,0.92}
\lstdefinestyle{mystyle}{
    backgroundcolor=\color{backcolour},   
    commentstyle=\color{codegreen},
    keywordstyle=\color{magenta},
    numberstyle=\tiny\color{codegray},
    stringstyle=\color{codepurple},
    basicstyle=\ttfamily\footnotesize,
    breakatwhitespace=false,         
    breaklines=true,                 
    captionpos=b,                    
    keepspaces=true,                 
    numbers=left,                    
    numbersep=5pt,                  
    showspaces=false,                
    showstringspaces=false,
    showtabs=false,                  
    tabsize=2
}

\lstset{style=mystyle}
\author{Oskar Idland}
\title{Lecture 11}
\date{}
\begin{document}
\maketitle
\newpage
\section*{2 spin - 1/2 system}
\[
\ket{t} = \frac{1}{\sqrt{2}}\left( \ket{↑} \otimes \ket{↓} + \ket{↓} \otimes  \ket{↑}\right)
\]
\paragraph{Probability to measure spin 1 to be $ℏ / 2$ along z-axis}
We define spin 1 to be when the first spin is up and the second is down. Spin 2 is when the first spin is down and the second is up. 
\[
P_t = \bra{t} \Big(\ket{↑}\bra{↑} ⊗  I\Big) \ket{t} = \frac{1}{2} \Big( \bra{↑} ⊗  \bra{↓} + \bra{↓} ⊗  \bra{↑}\Big) \underbrace{\Big(\ket{↑}\bra{↑} ⊗  I\Big) \Big( \ket{↑} ⊗  \ket{↓} + \ket{↓} ⊗  \ket{↑}\Big)}_{\ket{↑} ⊗ \ket{↓}}
\]
\[
P_t = \frac{1}{2} \Bigl( \bra{↑}\ket{↑} \bra{↓}\ket{↓} + \bra{↓}\ket{↑} \bra{↑}\ket{↓}\Bigr) = \frac{1}{2}
\]
\section*{Total Angular Momentum}
The total angular momentum is the sum of orbital and spin angular momentum
\[
\vec{J} = \vec{L} + \vec{S}
\]
This is actually represented on the form:
\[
\vec{J} = \vec{L} ⊗  I + I ⊗  \vec{S}
\]
\subsection*{Commutation}
\[
\left[\hat{L}, \hat{S}\right] = \left[\hat{L} ⊗ I, \hat{S} ⊗ I\right]
\]
\[
\left(L ⊗ I\right)\left(I ⊗ S\right) - \left(I ⊗ S\right) \left(L ⊗ I\right)
\]
\[
L ⊗ S - L ⊗ S = 0
\]
\subsection*{Total spin of two spin 1/2 particles}
\paragraph{Total spin of two particles spinning up}
\[
\vec{S}^{\text{tot}} = \vec{S} ⊗ I + I ⊗ \vec{S} 
\]
\[
\vec{S}^{\text{tot}} =S_z ⊗ I + I ⊗ S_z
\]
\[
S_z^{\text{tot}} \ket{↑} ⊗ \ket{↑}=  \Big(S_z ⊗ I + I ⊗ S_z\Big) \ket{↑} ⊗ \ket{↑} = \frac{ℏ}{2} \ket{↑} ⊗ \ket{↑} + \frac{ℏ}{2} \ket{↑} ⊗ \ket{↑} = ℏ \ket{↑} ⊗ \ket{↑}
\]
\paragraph{Total spin of one particle spinning up and one spinning down}
\[
S_z^{\text{tot}} \ket{↑} ⊗ \ket{↓} = \Big(S_z ⊗ I + I ⊗ S_z\Big) \ket{↑} ⊗ \ket{↓} = \frac{ℏ}{2} \ket{↑} ⊗ \ket{↓} + \frac{ℏ}{2} \ket{↑} ⊗ \ket{↓} = 0
\]
Naturally we se that the total spin is just the sum of the spin of each particle. 
Its easy then to see that
\[
S_z^{\text{tot}} \ket{↓} ⊗ \ket{↑} = 0 \quad , \quad S_z^{\text{tot}} \ket{↓} ⊗ \ket{↓} = - ℏ \ket{↓} ⊗ \ket{↓}
\]
\subsection*{Eigenstate of total spin $S^{\text{tot}}$ squared}
\[
\vec{S}^{\text{tot}^2} = \vec{S}^{\text{tot}} ⋅ \vec{S}^{\text{tot}}
\]
\[
\vec{S}^{\text{tot}^2} = \Bigl(\vec{S} ⊗ I + I ⊗ \vec{S}\Bigr) ⋅ \Bigl(\vec{S} ⊗ I + I ⊗ \vec{S}\Bigr)
\]
\[
\vec{S}^{\text{tot}^2}=  \vec{S}^2 ⊗ I + I ⊗ \vec{S}^2 + 2 \Bigl(S_x ⊗  S_x + S_y ⊗ S_y + S_z ⊗ S_z\Bigr)
\]
We must define $S_i$ in terms of the raising and lowering operators. 
\[
S_x ⊗ S_x = \frac{1}{4} \Bigl(S_+ + S_-\Bigr) ⊗ \Bigl(S_+ + S_-\Bigr)
\]
\[
S_y ⊗ S_y = \frac{1}{4} \Bigl(S_{+} S_-\Bigr) ⊗ \Bigl(S_+ + S_-\Bigr) 
\]
\[
S_x ⊗ S_x+  S_y ⊗ S_y = \frac{1}{2} \Bigl(S_+ ⊗  S_- + S_- ⊗ S_+\Bigr)
\]
We can now derive the total:
\[
S^{\text{tot}^2} = S^2 ⊗ I + I ⊗ S^2 + S_+ ⊗ S_- + S_- ⊗ S_+ + 2 S_z ⊗ S_z
\]

\paragraph{Eigenstate of total spin $S^{\text{tot}}$ squared for two particles spinning up}

\[
S^{\text{tot}^2} \ket{↑} ⊗ \ket{↑} = \Bigl(S^2 ⊗ I + I ⊗ S^2 + S_+ ⊗ S_- + S_- ⊗ S_+ + 2 S_z ⊗ S_z\Bigr) \ket{↑} ⊗ \ket{↑}
\]
\[
S^{\text{tot}^2} \ket{↑} ⊗ \ket{↑} = ℏ^2 \frac{3}{4} \ket{↑} ⊗ \ket{↑} + ℏ^2 \frac{3}{4} \ket{↑} ⊗ \ket{↑} + ℏ^2 \ket{\text{NULL}} ⊗ \ket{↓} + \ket{\text{NULL}} + 2 \left(\frac{ℏ}{2}\right)^2 \ket{↑} ⊗ \ket{↑} 
\]
\[
S^{\text{tot}^2} \ket{↑} ⊗ \ket{↑} = ℏ^2 \left(\frac{3}{4} + \frac{3}{4} + \frac{1}{2}\right) \ket{↑} ⊗ \ket{↑} = 2ℏ^2 \ket{↑} ⊗ \ket{↑}
\]
\paragraph{Eigenstate of total spin $S^{\text{tot}}$ of two particles spinning down}
\[
S^{\text{tot}^2} \ket{↓} ⊗ \ket{↓} = 2ℏ^2 \ket{↓} ⊗ \ket{↓}
\]
\paragraph{Eigenstate of total spin $S^{\text{tot}}$ of one particle spinning up and one spinning down}
\[
S^{\text{tot}^2} \ket{↑} ⊗ \ket{↓} = \Bigl(S^2 ⊗ I + I ⊗ S^2 + S_+ ⊗ S_- + S_ ⊗ S_+ + 2S_z ⊗ S_z\Bigr) \ket{↑} ⊗ \ket{↓}
\]
\[
ℏ^2 \frac{3}{4} \ket{↑} ⊗ \ket{↓} + h^2 \frac{3}{4} \ket{↑} ⊗ \ket{↓} + 0 + ℏ^2 \ket{↓} ⊗ \ket{↑} + 2 \frac{ℏ}{2} ⋅ \left(\frac{ℏ}{2}\right) \ket{↑} ⊗ \ket{↓}
\]
\[
ℏ^2 \Bigl(\ket{↑} ⊗ \ket{↓} + \ket{↓} ⊗ \ket{↑}\Bigr)
\]
\paragraph{Eigenstate of total spin $S^{\text{tot}}$ of one particle down up and one spinning up}
\[
\vec{S}^{\text{tot}^2} \ket{↓} ⊗ \ket{↑} = ℏ^2 \Big(\ket{↓} ⊗ \ket{↑} + \ket{↑} ⊗ \ket{↓}\Big)
\]



\paragraph{Conclusion: }
$\ket{↑} ⊗  \ket{↓}$ and $\ket{↓} ⊗ \ket{↑}$ are not eigenstates but $\ket{↑} ⊗ \ket{↑}$ and $\ket{↓} ⊗ \ket{↓}$ are.

\subsection*{Creating Linear Combinations}
\[
S^{\text{tot}^2} \Big(\ket{↑}⊗ \ket{↓} - \ket{↓} ⊗ \ket{↑}\Big) = 0
\]
This is an eigenstate. Also known as the singlet state

\[
S^{\text{tot}^2} \Big(\ket{↑}⊗ \ket{↓} + \ket{↓} ⊗ \ket{↑}\Big) = 2 ℏ^2 \Big(\ket{↑ }⊗ \ket{↓} + \ket{↓} ⊗ \ket{↑}\Big)
\]
This is one of the three parts of the triplet state. The other being $\ket{↑} ⊗ \ket{↑}$ and $\ket{↓} ⊗ \ket{↓}$

\section*{General Combinations of ang.mom. states}
\[
\ket{sm} = ∑_{m_1 m_2}^{} C_{m_1 m_2}^{S_1 S_2 S} \ket{s_1 m_1} ⊗ \ket{s_2 m_2}
\]
where $C$ is the Clebsch-Gordan coefficient.
\[
\ket{s_1 m_1} ⊗ \ket{s_2 m_2}=  ∑_{m_1 m_2}^{} C_{m_1 m_2}^{S_1 S_2 S} \ket{s m}
\]

\paragraph{a spin - 3 / 2 with a spin-1 to get a total spin 3/2, m = 3/2}
\[
\ket{3 / 2, + \ket{3 / 2}}
\]
Reading from the table we see the column of 3/2 and +3/2, having values 3/5 square root of 3/5 for m1 = +3/2 and s1 = 0. We get the negative of the square root of 2/5 for m1 = +1/2 and s1 = 1.
\[
\ket{\underbrace{3 / 2}_{s}, + \underbrace{\ket{3 / 2}}_{m}} = \sqrt{\frac{3}{5}} \ket{\underbrace{3 / 2}_{s_1}, + \ket{\underbrace{3 / 2}_{m_1}}} ⊗ \ket{\underbrace{1}_{s_2},\underbrace{0}_{m_2}} - \sqrt{\frac{2}{5}} \ket{\underbrace{3 / 2}_{s_1}, + \ket{\underbrace{1 / 2}_{m_1}}} ⊗ \ket {\underbrace{1}_{s_2},\underbrace{+1}_{m_2}} 
\]  
\end{document}