\documentclass{article}
\usepackage{amsmath}
\usepackage[mathletters]{ucs}
\usepackage[utf8x]{inputenc}
\usepackage[margin=1.5in]{geometry}
\usepackage{enumerate}
\newtheorem{theorem}{Theorem}
\usepackage[dvipsnames]{xcolor}
\usepackage{pgfplots}
\pgfplotsset{compat=1.18}
\setlength{\parindent}{0cm}
\usepackage{graphics}
\usepackage{graphicx} % Required for including images
\usepackage{subcaption}
\usepackage{bigintcalc}
\usepackage{pythonhighlight} %for pythonkode \begin{python}   \end{python}
\usepackage{appendix}
\usepackage{arydshln}
\usepackage{physics}
\usepackage{booktabs} 
\usepackage{adjustbox}
\usepackage{mdframed}
\usepackage{relsize}
\usepackage{physics}
\usepackage[thinc]{esdiff}
\usepackage{esint}  %for lukket-linje-integral
\usepackage{xfrac} %for sfrac
\usepackage{hyperref} %for linker, må ha med hypersetup
\usepackage[noabbrev, nameinlink]{cleveref} % to be loaded after hyperref
\usepackage{amssymb} %\mathbb{R} for reelle tall, \mathcal{B} for "matte"-font
\usepackage{listings} %for kode/lstlisting
\usepackage{verbatim}
\usepackage{graphicx,wrapfig,lipsum,caption} %for wrapping av bilder
\usepackage{mathtools} %for \abs{x}
\usepackage[norsk]{babel}
\usepackage{cancel}
% \usepackage{emoji}
\definecolor{codegreen}{rgb}{0,0.6,0}
\definecolor{codegray}{rgb}{0.5,0.5,0.5}
\definecolor{codepurple}{rgb}{0.58,0,0.82}
\definecolor{backcolour}{rgb}{0.95,0.95,0.92}
\lstdefinestyle{mystyle}{
    backgroundcolor=\color{backcolour},   
    commentstyle=\color{codegreen},
    keywordstyle=\color{magenta},
    numberstyle=\tiny\color{codegray},
    stringstyle=\color{codepurple},
    basicstyle=\ttfamily\footnotesize,
    breakatwhitespace=false,         
    breaklines=true,                 
    captionpos=b,                    
    keepspaces=true,                 
    numbers=left,                    
    numbersep=5pt,                  
    showspaces=false,                
    showstringspaces=false,
    showtabs=false,                  
    tabsize=2
}

\lstset{style=mystyle}
\author{Oskar Idland}
\title{Oblig 6}
\date{}
\begin{document}
\maketitle
\newpage

\section*{Problem 6.8 (H)}
\subsection*{a)}
There are two possible states with the same probability coefficients. We therefore know that the probability of being in either state is $\frac{1}{2}$. When spin number 1 is measured to be $+ ℏ / 2$. We are in the first state, which we know have a probability of 50\%. We can also calculate this as follows:
\[
\ket{ψ} = \frac{1}{\sqrt{2}} \Big(\ket{↑} ⊗ \ket{↓} - \ket{↓} ⊗ \ket{↑}\Big)
\]
\[
P(\ket{↑↓}) = \bra{ψ}\Big(\ket{↑} \bra{↑} ⊗ I\Big)\ket{ψ}
\]

\begin{align*}
P(\ket{↑↓}) &= \frac{1}{2} \Big(\bra{↑} ⊗ \bra{↓} - \bra{↓} ⊗ \bra{↑}\Big) \Big(\ket{↑} \bra{↑} ⊗ I\Big) \Big(\ket{↑} ⊗ \ket{↓} - \ket{↓}⊗ \ket{↑}\Big) \\
&= \frac{1}{2} \Big(\bra{↑} ⊗ \bra{↓} - \bra{↓} ⊗ \bra{↑}\Big) \Big(\ket{↑}\underbrace{\bra{↑}\ket{↑}}_{1} ⊗ I \ket{↓} - \ket{↑} \underbrace{\bra{↑}\ket{↓}}_{0} ⊗ I \ket{↑}\Big)\\
&= \frac{1}{2} \Big(\bra{↑} ⊗ \bra{↓} - \bra{↓} ⊗ \bra{↑}\Big) \Big(\ket{↑} ⊗ \ket{↓}\Big)  \\
&= \frac{1}{2} \Big(\bra{↑}\ket{↑} ⋅ \bra{↓}\ket{↓} - \bra{↓}\ket{↑} ⋅ \bra{↑}\ket{↓}\Big) \\
P(\ket{↑↓}) &= \frac{1}{2}
\end{align*}

\subsection*{b)}
As the the state vector $\ket{ψ}$ is in a superposition of two state, where in each, the spins are in opposite direction, we know there is 100\% probability of measuring opposite spins during measurement. We can also calculate this as follows:
\begin{align*}
    P(\text{opposite spins}) &= P(\ket{↑↓}) + P(\ket{↓↑}) \\
    P(\ket{↓↑}) &= \bra{ψ}\Big(\ket{↓} \bra{↓} ⊗ I\Big)\ket{ψ} \\
    &= \frac{1}{2} \Big(\bra{↑} ⊗ \bra{↓} - \bra{↓} ⊗ \bra{↑}\Big) \Big(\ket{↓} \bra{↓} ⊗ I\Big) \Big(\ket{↑} ⊗ \ket{↓} - \ket{↓}⊗ \ket{↑}\Big) \\
    &= \frac{1}{2} \Big(\bra{↑} ⊗ \bra{↓} - \bra{↓} ⊗ \bra{↑}\Big) \Big(\ket{↓} \underbrace{\bra{↓}\ket{↑}}_{0} ⊗ I \ket{↓} - \ket{↓} \underbrace{\bra{↓}\ket{↓}}_{1} ⊗ I \ket{↑}\Big) \\
    &= \frac{1}{2} \Big(\bra{↑} ⊗ \bra{↓} - \bra{↓} ⊗ \bra{↑}\Big) \Big(-\ket{↓} ⊗ \ket{↑}\Big)  \\
    &= \frac{1}{2} \Big(-\bra{↑}\ket{↓} ⋅ \bra{↓}\ket{↑} + \bra{↓}\ket{↓} ⋅ \bra{↑}\ket{↑}\Big) \\
    &= \frac{1}{2} \\
    P(\text{opposite spins}) &= \frac{1}{2} + \frac{1}{2} = 1
\end{align*}

\subsection*{c)}
We use the same method as in the previous task, and for the same reasons we know the probability of opposite spins is 100\%.:
\[
\ket{ψ_x} = \frac{1}{\sqrt{2}} \Big(\ket{↑_x} ⊗ \ket{↓_x} - \ket{↓_x} ⊗ \ket{↑_x}\Big)
\]
\begin{align*}
    P(\ket{↑_x↓_x}) &= \bra{ψ_x}\Big(\ket{↑_x} \bra{↑_x} ⊗ I\Big)\ket{ψ_x} \\
    &= \frac{1}{\sqrt{2}}\bra{ψ_x}\Big(\ket{↑_x}\bra{↑_x}⊗I\Big) \Big(\ket{↑_x} ⊗ \ket{↓_x} - \ket{↓_x} ⊗ \ket{↑_x}\Big) \\
    &= \frac{1}{\sqrt{2}} \bra{ψ_x} \Big(\ket{↑_x}\bra{↑_x}\ket{↑_x} ⊗ I \ket{↓_x} - \ket{↑_x}\bra{↑_x}\ket{↓_x} ⊗ I \ket{↑_x}\Big) \\
    &= \frac{1}{2} \Big(\bra{↑_x} ⊗ \bra{↓_x} - \bra{↓_x} ⊗ \bra{↑_x}\Big) \Big(\ket{↑_x} ⊗ \ket{↓_x}\Big) \\
    &= \frac{1}{2} \Big(\bra{↑_x}\ket{↑_x} ⋅ \bra{↓_x}\ket{↓_x} - \bra{↓_x}\ket{↑_x} ⋅ \bra{↑_x}\ket{↓_x}\Big) \\
    P(\ket{↑_x↓_x}) &= \frac{1}{2} \\
    P(\text{opposite spins}) &= P(\ket{↑_x↓_x}) + P(\ket{↓_x↑_x}) \\
    P(\text{opposite spins}) &= \frac{1}{2} + \frac{1}{2} = 1
\end{align*}

\subsection*{d)}
\begin{align*}
    \ket{ψ_x} &= \frac{1}{\sqrt{2}} \Big(\ket{↑_x} ⊗ \ket{↓_x} - \ket{↓_x} ⊗ \ket{↑_x}\Big) \\
    \ket{ψ_x} &= \frac{1}{\sqrt{2}} \Bigg(\frac{1}{\sqrt{2}}\Big(\ket{↑} + \ket{↓} \Big) ⊗ \frac{1}{\sqrt{2}} \Big(\ket{↑} - \ket{↓}\Big)- \frac{1}{\sqrt{2}}\Big(\ket{↑} - \ket{↓}\Big) ⊗ \frac{1}{\sqrt{2}} \Big(\ket{↑} + \ket{↓}\Big)\Bigg) \\
    \ket{ψ_x} &= \frac{1}{2\sqrt{2}} \Bigg( \Big(\ket{↑} + \ket{↓}\Big) ⊗ \Big(\ket{↑} - \ket{↓}\Big) - \Big( \ket{↑} ⊗ \ket{↑} + \ket{↑} ⊗  \ket{↓} - \ket{↓} ⊗ \ket{↑} - \ket{↓} ⊗ \ket{↓} \Big)\Bigg) \\
    \ket{ψ_x} &= \frac{1}{2\sqrt{2}} \Bigg( \cancel{\ket{↑} ⊗ \ket{↑}} - \ket{↑} ⊗ \ket{↓} + \ket{↓} ⊗ \ket{↑} - \cancel{\ket{↓} ⊗ \ket{↓}} - \cancel{\ket{↑} ⊗ \ket{↑}} - \ket{↑} ⊗ \ket{↓} + \ket{↓} ⊗ \ket{↑} \cancel{+ \ket{↓} ⊗ \ket{↓}} \Bigg) \\
    \ket{ψ_x} &= \frac{1}{2\sqrt{2}} \Bigg( -2\ket{↑} ⊗ \ket{↓} + 2\ket{↓} ⊗ \ket{↑} \Bigg) \\
    \ket{ψ_x} &= \frac{1}{\sqrt{2}} \Bigg( -\ket{↑} ⊗ \ket{↓} + \ket{↓} ⊗ \ket{↑} \Bigg) \\
    \ket{ψ_x} &= - \ket{ψ}
\end{align*}

As we found out that $\ket{ψ_x} = \ket{ψ}$ we have the same probabilities as in task a) and b). Therefore there is a 100\% probability of measuring opposite spins. We can express this as follows:
\[
P(\ket{↑↓}) = \bra{ψ_x}\Big(\ket{↑} \bra{↑} ⊗ I\Big)\ket{ψ_x} = \bra{ψ}\Big(\ket{↑} \bra{↑} ⊗ I\Big)\ket{ψ} = \frac{1}{2} 
\]
\[
P(\text{opposite spins}) = P(\ket{↑↓}) + P(\ket{↓↑}) = \frac{1}{2} + \frac{1}{2} = 1
\]

\subsection*{e)}
We know there is 100\% chance of measuring opposite spin along the same axis. Therefore, we know table III is incorrect, as measurement three, supposedly both gave spin down in the x-direction, which we know is impossible. The other tables are possible, as a measurement along the z-axis gives no restrictions on the possibles values of measurement along the x-axis, and vice versa.

\section*{Problem 6.9 (H)}
\subsection*{a)}
Just from looking at the coefficient of the state representing spin up, we know there is a $4 / 5$ chance of measuring spin up. The same can be done for the spin down state, which gives us a $1 / 5$ chance of measuring spin down. 

\subsection*{b)}
The values one can measure of $L^2$ is dependent on the possible $l$-values. In one of the possible states we have $l = 2$, and in the other $l = 1$. This gives us the following possible values of $L^2$:
\[
L^2_{l = 2} = ℏ^2 2(2 + 1) = 6 ℏ^2 \quad , \quad  L^2_{l = 1} = ℏ^2 1(1 + 1) = 2 ℏ^2
\]
The probabilities of measuring each value, is the same as the probability of being in each state. We therefore already know the probabilities:
\[
P(6ℏ^2) = \frac{1}{5} \quad , \quad P(2ℏ^2) = \frac{4}{5}
\]

For the possible values of $L_z$ we look at the possible $m$-values. In both states, $m=1$. This gives us the following possible values of $L_z$:
\[
L_{z_{m=1}} = ℏ m = ℏ 
\]
There is only one possible value which means the probability of measuring this value is 100\%:
\[
P(ℏ) = 1
\]

For the possible values of $S^2$ we must look at the possible $s$-values. As an electron is a spin-$1 / 2$ particle, we have $s = 1 / 2$. This gives only one value for $S^2 = 2ℏ^2$ with a 100\% probability.

\subsection*{c)}
To find the possible values of $\hat{J}^2$ we look at Clebsch-Gordan coefficients. When spin is up, we have $s = 1 / 2$ and $m_s = 1 / 2$. When spin is down, we have the opposite $m_s$. We therefore rewrite the state vector as follows:
\[
\ket{ψ} = \frac{1}{\sqrt{5}} \ket{3,2,1} ⊗ \ket{1 / 2, - 1 / 2} + \frac{2}{\sqrt{5}} \ket{3,2,1} ⊗ \ket{1 / 2, 1 / 2}
\]
We first compute the new ket for the first state:
\[
\ket{ψ_1} = \frac{1}{\sqrt{5}} \ket{3,2,1} ⊗ \ket{1 / 2, - 1 / 2} = \frac{1}{\sqrt{5}} \left(\sqrt{\frac{2}{5}} \ket{3,5 / 2, 1 / 2} + \sqrt{\frac{3}{5}} \ket{3, 3 / 2, 1 / 2}\right)
\]
Next for the second state:
\[
\ket{ψ_2} = \sqrt{\frac{4}{5}} \ket{3,1,1} ⊗ \ket{1 / 2, 1 / 2} = \sqrt{\frac{4}{5}}\ket{2,3 / 2, 3 / 2}
\]
The possible $j$-values are $\frac{5}{2}$ and $\frac{3}{2}$. Plugging this into the equation for $\hat{J}^2$ we get:
\[
\hat{J}^2\left(\frac{5}{2}\right) = \frac{35}{4}ℏ^2 \quad , \quad \hat{J}^2\left(\frac{3}{2}\right) = \frac{15}{4}ℏ^2 
\]
The probabilities of measuring each value is again proportional to their states respective coefficients. 
\[
P(3 / 2) = \frac{4}{5} + \frac{3}{25} = 23 / 25
\] 
\[
P(5 / 2) = 1 - P(3 / 2) = 2 / 25
\]

\subsection*{d}
The total spin $J_z$ is given by the quantum number $m_j$ which takes the values $1 / 2$ and $3 / 2$. We therefore only need to look at the coefficients of these states.
\[
P(m_j = 1 / 2) = \frac{2}{25} + \frac{3}{25} = \frac{1}{5}
\]
\[
P(m_j = 3 / 2) = 1 - P(m_j = 1 / 2) =  \frac{4}{5}
\]

To find the radial probability density in a state of $s_z = ℏ / 2$ we must integrate over all radial positions $r$. 
\[
P_{\ket{↑}}(r) = ∫_{0}^{π}  \ \mathrm{d}θ ∫_{0}^{2π}  \ \mathrm{d}ϕ ∫_{0}^{∞} \sinθ r^2 \left|ψ(r, ϕ, θ)\right|^2  \ \mathrm{d}r 
\]
I don't what to replace $ψ(r, ϕ, θ)$ with, please don't fail me.  

\section*{Problem 6.10 (X)}
\subsection*{a)}
\[
\left(b_xσ_x + b_zσ_z\right)^2 = b_x^2σ_x^2 + b_z^2σ_z^2 + b_xb_z(\underbrace{σ_xσ_z + σ_zσ_x}_{0})
\]
\[
b_x^2 
\begin{pmatrix}
    0 & 1 \\
    1 & 0
\end{pmatrix}
\begin{pmatrix}
    0 & 1 \\
    1 & 0
\end{pmatrix}
+ b_z^2
\begin{pmatrix}
    1 & 0 \\
    0 & -1
\end{pmatrix}
\begin{pmatrix}
    1 & 0 \\
0 & -1
\end{pmatrix}
\]
\[
b_x^2
\begin{pmatrix}
    1 & 0 \\
    0 & 1
\end{pmatrix}
+ b_z^2
\begin{pmatrix}
    1 & 0 \\
    0 & 1
\end{pmatrix}
\]
\[
(b_x^2 + b_z^2) I_{2}
\]

\subsection*{b)}
To find the probability of measuring $-ℏ / 2$ after some time $t$, we must apply the time evolution operator to the state vector, and then apply the $\ket{↓}$-state. We know the vector has been measured to be in a spin-up state. 
\[
P(- ℏ /2) = \Big|\bra{↓} e^{-iHt / ℏ} \ket{↑}\Big|^2
\]
Expanding the exponential and replacing $\hat{H}$ with the pauli matrices we get:
\[
e^{-i\hat{H}t / ℏ} = e^{i(b_xσ_x + b_zσ_z)t / ℏ} = I_{2} + i \frac{t}{ℏ}\left(b_xσ_x + b_z σ_z\right) - \frac{1}{2!}\left(\frac{t}{ℏ}\right)^2  \underbrace{\left(b_xσ_x + b_zσ_z\right)^2}_{(b_x^2 + b_z^2)I_2} + i \frac{1}{3!} \left(\frac{t}{ℏ}\right)^3 \left(b_xσ_x + b_zσ_z\right)^3 \ldots 
\]
The pattern can be used to write the exponential in terms of $\sin$ and $\cos$. 
\[
e^{i(b_xσ_x + b_zσ_z)t / ℏ} = I_2 \cos \left(\sqrt{b_x^2 + b_z^2}\frac{t}{ℏ}\right) + (b_xσ_x + b_zσ_z) \frac{i\sin \left(\sqrt{b_x^2 + b_z^2} \frac{t}{ℏ} \right)}{\sqrt{b_x^2 + b_z^2}} 
\]
As only the $σ_x$ can turn the $\ket{↑}$ to a $\ket{↓}$, and the to states being orthonormal, we are left with one term, men acting the operator on the ket. 
\[
P(- ℏ / 2) = \frac{b_x^2}{b_x^2 + b_z^2} \sin ^2 \left(\sqrt{b_x^2 + b_z^2} \frac{t}{ℏ}\right)
\]

\subsection*{c)}
There are only two possible values we can measure. If the probability of measuring $+ℏ / 2$ is $a$ and measuring $-ℏ / 2$ is $b$, where $a + b = 1$ we can express the expectation value to be the following. 
\[
\left<S_z(t)\right> = \frac{ℏ}{2} \left(1 - b\right) 
\]
If there is zeros percent chance of measuring spin down, then $b  = 0$, and the expectation value is $ℏ / 2$, and vice versa.


\section*{Problem 6.11 (X)}
\subsection*{a)}
Total angular momentum must be conserved. We begin with a system of just one proton with spin $1 / 2$. When the neutron becomes a proton and a electron, the total spin must still be $1 / 2$. For this to be possible, one of the particles must have negative orbital angular momentum. Lets say the proton is particle A, and the electron is particle B with a relative orbital angular momentum $L$. We then have the following:

\[
j = s_A + s_B + ℓ
\]
\[
\frac{1}{2} = \frac{1}{2} + \frac{1}{2} + ℓ
\]
\[
ℓ = -1
\]
$L$ cannot be negative, therefore, there is no possible way for the neutron to decay into a proton and an electron. 

\subsection*{b)}
We do the same setup again:
\[
j = s_A + s_B + ℓ
\]
\[
\frac{3}{2} = \frac{1}{2} + ℓ
\]
\[
ℓ = 1
\]
We know that the spin of particle $s$ can also be $- 1 / 2$. 
\[
\frac{3}{2} = -\frac{1}{2} + ℓ
\]
\[
ℓ = 2
\]
We see the particle $B$ can have relative orbital angular momentum $1$ and $2$. When $ℓ = 1$ we know the spin component in the $z$-direction must be $+ℏ / 2$. Therefore there is 100\% chance of measuring this value, if $ℓ = 1$, and 0\% chance if $ℓ = 2$.  
\end{document}