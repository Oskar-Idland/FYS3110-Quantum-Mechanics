\documentclass{article}
\usepackage{amsmath}
\usepackage[mathletters]{ucs}
\usepackage[utf8x]{inputenc}
\usepackage[margin=1.5in]{geometry}
\usepackage{enumerate}
\newtheorem{theorem}{Theorem}
\usepackage[dvipsnames]{xcolor}
\usepackage{pgfplots}
\setlength{\parindent}{0cm}
\usepackage{graphics}
\usepackage{graphicx} % Required for including images
\usepackage{subcaption}
\usepackage{bigintcalc}
\usepackage{pythonhighlight} %for pythonkode \begin{python}   \end{python}
\usepackage{appendix}
\usepackage{arydshln}
\usepackage{physics}
\usepackage{booktabs} 
\usepackage{adjustbox}
\usepackage{mdframed}
\usepackage{relsize}
\usepackage{physics}
\usepackage[thinc]{esdiff}
\usepackage{fixltx2e}
\usepackage{esint}  %for lukket-linje-integral
\usepackage{xfrac} %for sfrac
\usepackage{hyperref} %for linker, må ha med hypersetup
\usepackage[noabbrev, nameinlink]{cleveref} % to be loaded after hyperref
\usepackage{amssymb} %\mathbb{R} for reelle tall, \mathcal{B} for "matte"-font
\usepackage{listings} %for kode/lstlisting
\usepackage{verbatim}
\usepackage{graphicx,wrapfig,lipsum,caption} %for wrapping av bilder
\usepackage{mathtools} %for \abs{x}
\usepackage[norsk]{babel}
\usepackage{cancel}
\definecolor{codegreen}{rgb}{0,0.6,0}
\definecolor{codegray}{rgb}{0.5,0.5,0.5}
\definecolor{codepurple}{rgb}{0.58,0,0.82}
\definecolor{backcolour}{rgb}{0.95,0.95,0.92}
\lstdefinestyle{mystyle}{
    backgroundcolor=\color{backcolour},   
    commentstyle=\color{codegreen},
    keywordstyle=\color{magenta},
    numberstyle=\tiny\color{codegray},
    stringstyle=\color{codepurple},
    basicstyle=\ttfamily\footnotesize,
    breakatwhitespace=false,         
    breaklines=true,                 
    captionpos=b,                    
    keepspaces=true,                 
    numbers=left,                    
    numbersep=5pt,                  
    showspaces=false,                
    showstringspaces=false,
    showtabs=false,                  
    tabsize=2
}

\lstset{style=mystyle}
\author{Oskar Idland}
\title{Problem  Set 2}
\date{}
\begin{document}
\maketitle
\newpage

\section*{Problem 2.1 (L)}
\[
\ket{ψ} = ∑_{i=1}^{2} ψ_i \ket{a_i} = ∑_{i=1}^{2} ψ_i' \ket{a_i'}
\]

As the two bases are orthonormal, we can express $ψ_i$ in the following manner:

\[
∑_{i=1}^2 ψ_i = ∑_{i=1}^2 \bra{a_i} ψ_i' \ket{a_i'}  
\]

We can then express $ψ_i'$ in terms of $ψ_i$:

\[
∑_{i=1}^{2} ψ_i' = ∑_{i=1}^{2} \bra{a_i'} ψ_i \ket{a_i}
\]


\section*{Problem 2.2 (L)}
We begin by creating a matrix to convert from the basis  $B =  \left\{\begin{pmatrix*}[r]
 1 \\
 0 \\
\end{pmatrix*}, \begin{pmatrix*}[r]
 0 \\
 1 \\
\end{pmatrix*}\right\}$ into the new basis of $B' = \left\{\frac{1}{\sqrt{2}} \begin{pmatrix*}[r]
 1 \\
 1 \\
\end{pmatrix*}, \frac{1}{\sqrt{2}} \begin{pmatrix*}[r]
 -1 \\
 1 \\
\end{pmatrix*}\right\}$. By getting the combined matrix of the basis vectors into row reduced echelon form.  
\[
\begin{bmatrix*}[r]
 1 & 0 & \frac{1}{\sqrt{2}} & \frac{1}{\sqrt{2}} \\
 0 & 1 & \frac{1}{\sqrt{2}} & -\frac{1}{\sqrt{2}} \\
\end{bmatrix*}
\]
The last two columns creates the transformation matrix $T$. 
\[
\hat{O}' = T \hat{O}
\]


\section*{Problem 2.3 (L)}
We can see that $\hat{O}^{†} = \hat{O}$


\section*{Problem 2.4 (L)}
When an operator acts on its eigenstate, it returns the eigenvalue times the eigenstate.
\[
∴ \bra{λ}\hat{K}\ket{u} = λ\bra{λ}\ket{u}
\]

\section*{Problem 2.5 (L)}


\end{document}