\documentclass{article}
\usepackage{amsmath}
\usepackage[mathletters]{ucs}
\usepackage[utf8x]{inputenc}
\usepackage[margin=1.5in]{geometry}
\usepackage{enumerate}
\newtheorem{theorem}{Theorem}
\usepackage[dvipsnames]{xcolor}
\usepackage{pgfplots}
\pgfplotsset{compat=1.18}
\setlength{\parindent}{0cm}
\usepackage{graphics}
\usepackage{graphicx} % Required for including images
\usepackage{subcaption}
\usepackage{bigintcalc}
\usepackage{pythonhighlight} %for pythonkode \begin{python}   \end{python}
\usepackage{appendix}
\usepackage{arydshln}
\usepackage{physics}
\usepackage{booktabs} 
\usepackage{adjustbox}
\usepackage{mdframed}
\usepackage{relsize}
\usepackage{physics}
\usepackage[thinc]{esdiff}
\usepackage{esint}  %for lukket-linje-integral
\usepackage{xfrac} %for sfrac
\usepackage{hyperref} %for linker, må ha med hypersetup
\usepackage[noabbrev, nameinlink]{cleveref} % to be loaded after hyperref
\usepackage{amssymb} %\mathbb{R} for reelle tall, \mathcal{B} for "matte"-font
\usepackage{listings} %for kode/lstlisting
\usepackage{verbatim}
\usepackage{graphicx,wrapfig,lipsum,caption} %for wrapping av bilder
\usepackage{mathtools} %for \abs{x}
\usepackage[norsk]{babel}
\usepackage{cancel}
\definecolor{codegreen}{rgb}{0,0.6,0}
\definecolor{codegray}{rgb}{0.5,0.5,0.5}
\definecolor{codepurple}{rgb}{0.58,0,0.82}
\definecolor{backcolour}{rgb}{0.95,0.95,0.92}
\lstdefinestyle{mystyle}{
    backgroundcolor=\color{backcolour},   
    commentstyle=\color{codegreen},
    keywordstyle=\color{magenta},
    numberstyle=\tiny\color{codegray},
    stringstyle=\color{codepurple},
    basicstyle=\ttfamily\footnotesize,
    breakatwhitespace=false,         
    breaklines=true,                 
    captionpos=b,                    
    keepspaces=true,                 
    numbers=left,                    
    numbersep=5pt,                  
    showspaces=false,                
    showstringspaces=false,
    showtabs=false,                  
    tabsize=2
}

\lstset{style=mystyle}
\author{}
\title{Problem  Set 2}
\date{}
\begin{document}
\maketitle
\newpage

\section*{Problem 2.1 (L)}
\[
\ket{ψ} = ∑_{i=1}^{2} ψ_i \ket{a_i} = ∑_{i=1}^{2} ψ_i' \ket{a_i'}
\]

We can express $ψ_i$ in the following manner:

\[
ψ_i = ∑_{i=1}^2 \bra{a_i} ψ_i' \ket{a_i'} = \underline{\underline{∑_{i=1}^{2} ψ_i' \bra{a_i} \ket{a_i'}}}
\]

We can then express $ψ_i'$ in terms of $ψ_i$:

\[
 ψ_i' = \underline{\underline{∑_{i=1}^{2} ψ_i\bra{a_i'} \ket{a_i}}}
\]


\section*{Problem 2.2 (L)}

We begin by creating a matrix $T$ to convert from the basis  $B =  \left\{\begin{pmatrix*}[r]
 1 \\
 0 \\
\end{pmatrix*}, \begin{pmatrix*}[r]
 0 \\
 1 \\
\end{pmatrix*}\right\}$ 
into the new basis of $B' = \left\{\frac{1}{\sqrt{2}} \begin{pmatrix*}[r]
 1 \\
 1 \\
\end{pmatrix*}, \frac{1}{\sqrt{2}} \begin{pmatrix*}[r]
 -1 \\
 1 \\
\end{pmatrix*}\right\}$.
By using the transformation matrix $T$ given by: 
\[
T = 
\begin{bmatrix*}[r]
 \frac{1}{\sqrt{2}} & \frac{1}{\sqrt{2}}  \\
 \frac{1}{\sqrt{2}} & -\frac{1}{\sqrt{2}}  \\
\end{bmatrix*}
\]
We also need the inverse matrix $T^{-1}$ to transform from our basis $B$ into the other basis $B'$. 
\[
\begin{bmatrix*}[r]
    \frac{1}{\sqrt{2}} & \frac{1}{\sqrt{2}} & 1 & 0 \\
    \frac{1}{\sqrt{2}} & -\frac{1}{\sqrt{2}} & 0 & 1 \\
\end{bmatrix*} ∼ 
\begin{bmatrix*}[r]
 1 & 0 & \frac{1}{\sqrt{2}} & \frac{1}{\sqrt{2}} \\
 0 & 1 & \frac{1}{\sqrt{2}} & -\frac{1}{\sqrt{2}} \\
\end{bmatrix*}
\]
The last two columns of the matrix forms the inverse matrix $T^{-1}$.
\[
T^{-1} = 
\begin{bmatrix*}[r]
    \frac{1}{\sqrt{2}} & \frac{1}{\sqrt{2}} \\
    \frac{1}{\sqrt{2}} & -\frac{1}{\sqrt{2}} \\
\end{bmatrix*}
\]
This gives us the operator $\hat{O}'$.
\[
\hat{O}' = T^{-1} \hat{O} T = 
\begin{bmatrix*}[r]
    \frac{1}{\sqrt{2}} & \frac{1}{\sqrt{2}} \\
    \frac{1}{\sqrt{2}} & -\frac{1}{\sqrt{2}} \\
\end{bmatrix*}
\begin{bmatrix*}[r]
 0 & -i \\
 i & 0 \\
\end{bmatrix*}  
\begin{bmatrix*}[r]
    \frac{1}{\sqrt{2}} & \frac{1}{\sqrt{2}} \\
    \frac{1}{\sqrt{2}} & -\frac{1}{\sqrt{2}} \\
\end{bmatrix*} = 
\underline{\underline{\begin{bmatrix*}[r]
 0 & i \\  
 -i & 0 \\  
\end{bmatrix*} = 
-\hat{O} = \hat{O}^{T}}}
\]

\section*{Problem 2.3 (L)}
\[
\bra{w}\hat{O}^{†}\ket{w} = \bra{w}\hat{O}\ket{w}^{*}
\]
We define $\ket{w} = \ket{u} + \ket{v}$. 
\[
\left(\bra{u} + \bra{v}\right) \hat{O}^{†} \left(\ket{u} + \ket{v}\right) = \left(\bra{u} + \bra{v}\right) \hat{O} \left(\ket{u} + \ket{v}\right)^{*}
\]


\section*{Problem 2.4 (L)}
When an operator acts on its eigenstate, it returns the eigenvalue times the eigenstate.
\[
\hat{K}\ket{λ} = λ\ket{λ}
\]
The eigenvalues are real and the operator Hermitian, so the following must be true if we conjugate both sides:
\[
\bra{u}\hat{K}\ket{λ} = λ\bra{u}\ket{λ}
\]
\[
\left(\bra{u}\hat{K}\ket{λ}\right)^{†} = \left(λ\bra{u}\ket{λ}\right)^{†}
\]
\[
\bra{λ}\hat{K}^{†}\ket{u} = λ^{*}\bra{λ}\ket{u}
\]
\[
∴ \bra{λ}\hat{K}\ket{u} = λ \bra{λ}\ket{u}
\]

\section*{Problem 2.5 (L)}
\[
\hat{O} = \ket{1}\bra{1} - \ket{2}\bra{2}
\]

\section*{Problem 2.6 (X)}
An Hermitian operator is defined as an operator which is equal to its Hermitian conjugate, meaning $\hat{L} = \hat{L}^{†}$. Hermitian operators have real eigenvalues. This is a good attribute for an operator which represent an observable, as the we only measure real values during experiments.
\paragraph{Proof of Real Eigenvalues: }

\[
\mathbf{L}\ket{λ} = λ\ket{λ}
\]
\[
\bra{λ}\mathbf{L}^{†} = \bra{λ}λ^*
\]
Multiplying the first equation with $\bra{λ}$ and the second with $\ket{λ}$
\[
\bra{λ}\mathbf{L}\ket{λ} = λ \bra{λ}\ket{λ}
\]
\[
\bra{λ}\mathbf{L}^{†}\ket{λ} = λ* \bra{λ}\ket{λ}
\]
As $\mathbf{L} = \mathbf{L}^{†}$, it follows that $λ = λ^*$ and it must therefore be real

\section*{Problem 2.7 (X)}
\[
\hat{O} = \hat{A} \hat{B}
\]
\[
\hat{A} \hat{B} = A_{ij} B_{i'j'} = AB
\]
We begin with a simple case of both $\hat{A}$ and $\hat{B}$ being $2×2$ matrices. 
\[
\hat{O} = \hat{A} \hat{B} = 
\begin{bmatrix} 
    A_{11} & A_{12} \\
    A_{21} & A_{22} \\ 
\end{bmatrix} 
\begin{bmatrix} 
    B_{11} & B_{12} \\
    B_{21} & B_{22} \\
\end{bmatrix} = 
\begin{bmatrix*}[r]
 A_{11}B_{11} + A_{12}B_{21} & A_{11}B_{12} + A_{12}B_{22} \\
 A_{21}B_{11} + A_{22}B_{21} & A_{21}B_{12} + A_{22}B_{22} \\
\end{bmatrix*}
\]
Lets check if the following equation holds. 
\[
\hat{O}^{†} \overset{\underset{\mathrm{?}}{}}{=} \hat{B}^{†} \hat{A}^{†}
\]

\subsection*{Sol. 1}
We first examine the left side
\[
\hat{O}^{†} = 
\begin{bmatrix*}[r]
A_{11}^{*}B_{11}^{*} + A_{12}^{*}B_{21}^{*} & A_{21}^{*}B_{11}^{*} + A_{22}^{*}B_{21}^{*} \\
A_{11}^{*}B_{12}^{*} + A_{12}^{*}B_{22}^{*} & A_{21}^{*}B_{12}^{*} + A_{22}^{*}B_{22}^{*} \\
\end{bmatrix*}
\]
Now the right side
\[
\hat{A}^{†} \hat{B}^{†} = 
\begin{bmatrix} 
A_{11} & A_{12} \\
A_{21} & A_{22} \\ 
\end{bmatrix}^{†}
\begin{bmatrix} 
B_{11} & B_{12} \\
B_{21} & B_{22} \\
\end{bmatrix}^{†} =
\begin{bmatrix} 
A_{11}^{*} & A_{21}^{*} \\
A_{12}^{*} & A_{22}^{*} \\
\end{bmatrix}
\begin{bmatrix*}[r]
B_{11}^{*} & B_{21}^{*} \\
B_{12}^{*} & B_{22}^{*} \\
\end{bmatrix*} 
\]
\[
\begin{bmatrix} 
A_{11}^{*} & A_{21}^{*} \\
A_{12}^{*} & A_{22}^{*} \\
\end{bmatrix}
\begin{bmatrix*}[r]
B_{11}^{*} & B_{21}^{*} \\
B_{12}^{*} & B_{22}^{*} \\
\end{bmatrix*} = 
\begin{bmatrix*}[r]
A_{11}^{*}B_{11}^{*} + A_{12}^{*}B_{21}^{*} & A_{21}^{*}B_{11}^{*} + A_{22}^{*}B_{21}^{*} \\
A_{11}^{*}B_{12}^{*} + A_{12}^{*}B_{22}^{*} & A_{21}^{*}B_{12}^{*} + A_{22}^{*}B_{22}^{*} \\
\end{bmatrix*}
\]
We can clearly see that
\[
\hat{O}^{†} = \hat{A}^{†} \hat{B}^{†}
\]
And how this generalizes to any $n×n$ matrix.

\subsection*{Sol. 2}
Each element in $\hat{O}$ can be written as
\[
O_{ij} = ∑_{k}^{N} A_{ik} B_{kj}
\]
We can then take the Hermitian conjugate as we know how it operates on numbers. 
\[
O_{ij}^{†} = ∑_{k}^{N} \left(A_{ik} B_{kj}\right)^{†}
\]
\[
O_{ji}^{*} = ∑_{k}^{N} B_{jk}^{*} A_{ki}^{*}
\]
Therefore we know that $\hat{O}^{†} = \hat{A}^{†} \hat{B}^{†}$.



\section*{Problem 2.8 (X)}
To find the Hermitian conjugate of an operator we must find an operator $K^{†}$ which satisfies the following:
\[
\bra{u}K\ket{v} = \bra{v}K^{†}\ket{u}
\]
Where $u$ and $v$ can be represented as functions $u(x)$ and $v(X)$. 
\[
\bra{u}K\ket{v} = ∫_{-∞}^{∞} u^{*}(x) x \frac{\mathrm{d}}{\mathrm{d}x} v(x) \ \mathrm{d}x
\]
We use integration by parts. 
\paragraph{Definition:}
\[
u = u^{*}(x)x \quad , \quad v' = \frac{\mathrm{d}}{\mathrm{d}x} v(x)
\]
\[
u' = x \frac{\mathrm{d}}{\mathrm{d}x} u^{*}(x) + u^{*}(x) \quad , \quad v = v(x)
\]

\[
\underbrace{u^{*}(x)\ x\ v(x) \Big|_{-∞}^{∞}}_{0} - ∫_{-∞}^{∞} \frac{\mathrm{d}}{\mathrm{d}x} (u^{*}(x)x) v(x) \ \mathrm{d}x 
\]
\[
- ∫_{-∞}^{∞} \left(\frac{\mathrm{d}}{\mathrm{d}x}u^{*}(x)x + u^{*}(x)\right)v(x) \ \mathrm{d}x
\]
\[
- ∫_{-∞}^{∞} \left(\frac{\mathrm{d}}{\mathrm{d}x}u^{*}(x)\left(x + 1\right)\right) v(x) \ \mathrm{d}x
\]
\[
- ∫_{-∞}^{∞} \left(x \frac{\mathrm{d}}{\mathrm{d}x}u^{*}(x) + \frac{\mathrm{d}}{\mathrm{d}x}u^{*}(x)\right)v(x) \ \mathrm{d}x
\]

\[
\underline{\underline{K^{†} = - x \frac{\mathrm{d}}{\mathrm{d}x} -1}}
\]


\section*{Problem 2.9 (E)}
\subsection*{a)}
By definition we know $ \bra{ψ} = \ket{ψ}^{†}$. 
\[
∴ \bra{ψ} = c^{*}\left(\sqrt{3} \bra{0} - i \bra{1}\right)
\]
To normalize this we must find $c$ such that $\bra{ψ}\ket{ψ} = 1$.
\[
\bra{ψ}\ket{ψ} = c^2 \left(3 \bra{0}\ket{0} - \bra{1}\ket{1}\right) = 1
\]
\[
2c^2 = 1 \quad , \quad c = \frac{1}{\sqrt{2}}
\]
The normalized $\ket{ψ}$ becomes
\[
\underline{\underline{\ket{ψ} = \frac{1}{\sqrt{2}}\left(\sqrt{3} \ket{0} + i \ket{1}\right)}}
\]
\subsection*{b)}
\[
\underline{\underline{\ket{ψ} ≃ 
\begin{pmatrix*}[r]
\sqrt{\frac{3}{2}} \\  
 \frac{i}{\sqrt{2}} \\  
\end{pmatrix*}}}
\]
\[
\hat{A} \ket{0} = -i \ket{1}
\]
\[
\begin{pmatrix*}[r]
 A_{11} & A_{12} \\
 A_{21} & A_{22} \\
\end{pmatrix*}
\begin{pmatrix*}[r]
 1 \\
 0 \\
\end{pmatrix*} = 
\begin{pmatrix*}[r]
0 \\
-i \\
\end{pmatrix*}
\]
This gives two equations. 
\[
A_{11} + 0 ⋅  A_{12} = 0 → A_{11} = 0
\]
\[
A_{21} + 0 ⋅ A_{22} = -i → A_{21} = -i
\]
\[
\underline{\hat{A} ≃ 
\begin{pmatrix*}[r]
 0 & A_{12} \\ 
 -i & A_{22} \\ 
\end{pmatrix*}}
\]
\[
\hat{A} \ket{1} =  i\ket{0}
\]
\[
\begin{pmatrix*}[r]
0 & A_{12} \\
-i & A_{22} \\
\end{pmatrix*}
\begin{pmatrix*}[r]
0 \\
1 \\
\end{pmatrix*} = 
\begin{pmatrix*}[r]
i \\
0 \\
\end{pmatrix*}
\]
Again, two new equations. 
\[
A_{12} = i
\]
\[
A_{22} = 0
\]
\[
\underline{\underline{\hat{A} ≃ 
\begin{pmatrix*}[r]
 0 & i \\ 
 -i & 0 \\
\end{pmatrix*}}}
\]

\subsection*{c)}
\[
\bra{ψ}\hat{A}\ket{ψ} = \frac{1}{2} 
\begin{pmatrix*}[r]
 \sqrt{3} & -i \\
\end{pmatrix*}
\begin{pmatrix*}[r]
 0 & i \\ 
 -i & 0 \\
\end{pmatrix*}
\begin{pmatrix*}[r]
 \sqrt{3} \\ 
 i \\
\end{pmatrix*}
\]
\[
\begin{pmatrix*}[r]
\sqrt{3} & -i \\
\end{pmatrix*}
\begin{pmatrix*}[r]
-1 \\ 
-i\sqrt{3} \\
\end{pmatrix*} = \underline{\underline{- 2\sqrt{3}}}
\]
\newline 
\[
\bra{ψ}\hat{A}\ket{ψ} = \left(\sqrt{3} \bra{0} - i \bra{1}\right) \hat{A} \left(\sqrt{3} \ket{0} + i\ket{1}\right)
\]
\[
\left(\sqrt{3} \bra{0} - i \bra{1}\right) \left(-i\sqrt{3}\ket{1} -\ket{0}\right)
\]
\[
- \sqrt{3} \bra{0}\ket{0} + ii \sqrt{3}\bra{1}\ket{1}
\]
\[
\underline{\underline{-2\sqrt{3}}}
\]


\section*{Problem 2.10 (E)}
\[
U = 
\begin{pmatrix*}[r]
 a & b \\
 c & d \\
\end{pmatrix*}
\]
\subsection*{a)}
\[
    U^{T} = 
\begin{pmatrix*}[r]
 a & c \\
 b & d \\
\end{pmatrix*}
\]
\[
    U^{†} = 
    \begin{pmatrix*}[r]
        a^{*} & c^{*} \\
 b^{*} & d^{*} \\
\end{pmatrix*}
\]

\subsection*{b)}
To be Hermitian, $U$ must be equal to its Hermitian conjugate.
\[
    U = U^{†}
    \]
    \[
        \begin{pmatrix*}[r]
a & b \\
c & d \\
\end{pmatrix*} = 
\begin{pmatrix*}[r]
a^{*} & c^{*} \\
b^{*} & d^{*} \\
\end{pmatrix*}
\]
Both $a$ and $d$ must be real, and $b = c^{*}$. 

\subsection*{c)}
We assume that $U$ has an eigenvector $\ket{u}$. 
\[
    U\ket{u} = λ\ket{u} \quad , \quad \ket{u} = 
\begin{pmatrix*}[r]
 u_1 \\
 u_2 \\
\end{pmatrix*}
\]
\[
\begin{pmatrix*}[r]
    a & b \\
 c & d \\
\end{pmatrix*}
\begin{pmatrix*}[r]
 u_1 \\
 u_2 \\
\end{pmatrix*} = λ \begin{pmatrix*}[r]
 u_1 \\
 u_2 \\
\end{pmatrix*}
\]
\[
\begin{pmatrix*}[r]
 au_1 + bu_2 \\
 cu_1 + du_2 \\
\end{pmatrix*} = λ 
\begin{pmatrix*}[r]
    u_1 \\
    u_2 \\
\end{pmatrix*}
\]
This gives us two equations 
\[
au_1 + bu_2 = λu_1
\]
\[
cu_1 + du_2 = λu_2
\]
\[
λ = \frac{au_1 + bu_2}{u_1} = \frac{cu_1 + du_2}{u_2}
\]
\[
(au_1 + bu_2)u_2 = (cu_1 + du_2)u_1
\]
\[
au_1u_2 + bu_2^2 - cu_1^2 - du_1u_2 = 0
\]
If this is to be true, and we take into account the demands on the elements of $U$ from b), we get the following:
\[
(au_1u_2 + bu_2^2 - cu_1^2 - du_1u_2) = (au_1u_2 + bu_2^2 - cu_1^2 - du_1u_2)^{*} = 0
\]
\[
(au_1u_2 + bu_2^2 - cu_1^2 - du_1u_2) - (au_1^{*}u_2^{*} + b^{*}u_2^2 - c^{*}u_1^{2} - du_1^{*}u_2^{*}) = 0
\]


\section*{Problem 2.11 (H)}
Both $\ket{ψ}$ and $\ket{ϕ}$ must both have real eigenvalues.  


\end{document}